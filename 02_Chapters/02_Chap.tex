% -*-coding: iso-latin-1  -*-
\chapter{Definici�n  del Espectro}

\begin{resumen}
En esta secci�n se presentan el  formalismo matem�tico \ldots. 
In this section we present the mathematical formalism for investigating the
spectrum of cardiac signals. Firstly, we introduce the lead-field bioelectric
model of cardiac sources and signals. Then, based on multivariate signal
analysis, we define the autocorrelation and the spectrum of cardiac sources.

Throughout this paper the following notation is used: $\langle\cdot\rangle_t$
denotes time-average, $\mathcal{F}[\cdot]$ is the Fourier Transform operator,
$(\ast)$ denotes convolution and $\delta(\cdot)$ is the Dirac's delta. We use
the following vector definitions: $\mathbf{1}=[1, 1, 1]^T$ and $\mathbf{0}=[0,
0, 0]^T$.
\end{resumen}


\section{Introducci�n}
\subsection{Se�al}
\subsection{Vector}
\subsection{Campo Vectorial}
\section{Autocorrelaci�n}

\subsection{Autocorrelation and spectrum of cardiac sources}

 The autocorrelation of a cardiac source, $\boldsymbol{\rho}(v,w,\tau)$,
 $\forall v,w\in V$, is defined as the collection of the cross-correlations
 between all pairs of dipoles in $V$. Since cardiac dipoles are vectorial
 entities, the cross-correlation between two dipoles consists of the
 cross-correlations between all three components of each dipole
 \cite{Papoulis91}.
 In order to define the autocorrelation of a cardiac source, the average dipole
 field $\bar{\mathbf{J}}(v)$ needs to be introduced:
 
 \begin{eqnarray}\label{eq:average_dipole}
 \bar{\mathbf{J}}(v)&=&\langle\mathbf{J}(v,t)\rangle_t  \nonumber \\
			     &=& [\langle J_x(v,t)\rangle_t, \langle J_y(v,t)\rangle_t, \langle J_z(v,t)\rangle_t]^T. 
 \end{eqnarray}
 Based on $\bar{\mathbf{J}}(v)$, we define the zero-average dipole field
 $\mathbf{J'}(v,t)=\mathbf{J}(v,t)-\bar{\mathbf{J}}(v)$, so that
 $\langle\mathbf{J'}(v,t)\rangle_t=\mathbf{0}$. The cross-correlation matrix
 between two cardiac dipoles $\mathbf{J}(v,t)$ and $\mathbf{J}(w,t)$, where
 $v,w\in V$, is then defined as
 
 \begin{eqnarray}\label{eq:autocorrelation_source}
 &&\boldsymbol{\rho}(v,w,\tau)=\langle\mathbf{J'}(v,t+\tau)\mathbf{J'}^{T}(w,t)\rangle_t
 \nonumber \\
 &&= \left( \begin{array}{ccc}
 {\rho}_{xx}(v,w,\tau) & {\rho}_{xy}(v,w,\tau) & {\rho}_{xz}(v,w,\tau) \\
 {\rho}_{yx}(v,w,\tau) & {\rho}_{yy}(v,w,\tau) & {\rho}_{yz}(v,w,\tau) \\
 {\rho}_{zx}(v,w,\tau) & {\rho}_{zy}(v,w,\tau) & {\rho}_{zz}(v,w,\tau) 
 \end{array} \right).
 \end{eqnarray}
 Therefore, each entry of $\boldsymbol{\rho}(v,w,\tau)$ contains the
 cross-correlation between one component of $\mathbf{J}(v,t)$ and one component
 of $\mathbf{J}(w,t)$. For instance, matrix entry ${\rho}_{zy}(v,w,\tau)$ is
 $\langle J'_z(v,t+\tau) J'_y(w,t)\rangle_t$. Also, the average power of dipole
 component  $J'_x(v,t)$ is by definition $P_x(v)={\rho}_{xx}(v,v,0)$, and
 analogous expressions can be obtained for the average power of dipole
 components $J'_y(v,t)$ and $J'_z(v,t)$.

 The spectrum of a cardiac source, $\boldsymbol{\sigma}(v,w,f)$, $\forall v,w\in
 V$, corresponds to the collection of the cross-spectra between all pairs of
 dipoles in $V$, and is defined as
 \begin{eqnarray}\label{eq:spectrum_source}
 &&\boldsymbol{\sigma}(v,w,f) =\mathcal{F}[\boldsymbol{\rho}(v,w,\tau)]
 \nonumber \\
 &&=\left( \begin{array}{ccc}
 {\sigma}_{xx}(v,w, f) & {\sigma}_{xy}(v,w, f) & {\sigma}_{xz}(v,w, f) \\
 {\sigma}_{yx}(v,w, f) & {\sigma}_{yy}(v,w, f) & {\sigma}_{yz}(v,w, f) \\
 {\sigma}_{zx}(v,w, f) & {\sigma}_{zy}(v,w, f) & {\sigma}_{zz}(v,w, f)
 \end{array} \right)
 \end{eqnarray}
 where the operator $\mathcal{F}[\cdot]$ is applied to
 $\boldsymbol{\rho}(v,w,\tau)$ on a component-by-component basis. For instance,
 ${\sigma}_{zy}(v,w, f)$ is $\mathcal{F}[{\rho}_{zy}(v,w,\tau)]$. 
 
 We also define the \emph{total cross-correlation} $R_{J}(v,w,\tau)$ between two
 cardiac dipoles $\mathbf{J}(v,t)$ and $\mathbf{J}(w,t)$ as the sum of the
 entries of $\boldsymbol{\rho}(v,w,\tau)$ and the \emph{total cross-spectrum}
 $S_{J}(v,w,f)$ as the Fourier Transform of $R_{J}(v,w,\tau)$. Mathematically,
 they can be expressed as
 
 \begin{eqnarray}
  R_{J}(v,w,\tau) &=&  \mathbf{1}^{T} \boldsymbol{\rho}(v,w,\tau) 
  \mathbf{1},\label{total_corr}\\
 S_{J}(v,w,f)  &=&  \mathbf{1}^{T} \boldsymbol{\sigma}(v,w, f)  \mathbf{1}.
\label{total_spectrum}
\end{eqnarray}

 Finally, we define the \emph{normalized} cross-correlation
$\boldsymbol{\hat{\rho}}(v,w,\tau)$ as the matrix of entries
\begin{equation}\label{eq:normalized_autocorrelation_source}
\hat{\rho}_{ab}(v,w,\tau) = \frac{{\rho}_{ab}(v,w,\tau)}{\sqrt{{\rho}_{aa}(v,v,0){\rho}_{bb}(w,w,0)}}
\end{equation}
 where $a,b\in \{x,y,z\}$, and the \emph{normalized} total cross-correlation
$\hat{R}_{J}(v,w,\tau)$ as
\begin{equation}\label{normalized_total_corr}
\hat{R}_{J}(v,w,\tau)= \frac{R_{J}(v,w,\tau)}{{\underset{\tau}{\max}}\{R_{J}(v,w,\tau)\}}.
\end{equation}



\subsection{Autocorrelation and spectrum of cardiac signals}

 Let $c(t)$ be a cardiac signal measured by applying $\mathbf{L}(v)$ to a
cardiac source of autocorrelation $\boldsymbol{\rho}(v,w,\tau)$ and spectrum $\boldsymbol{\sigma}(v,w,f)$ in $V$. Define $c'(t)$ as the cardiac signal $c(t)$ minus its time-average value $\bar{c}=\langle c(t) \rangle_t$,
\begin{equation}\label{c_minus_average}
c'(t)=c(t)-\bar{c}.
\end{equation}
 The autocorrelation function $R_{c}(\tau)$ of the cardiac signal $c(t)$ is
defined as the following average \cite{Papoulis91}:
\begin{equation}\label{signal_autocorr}
R_{c}(\tau)  = \langle c'(t+\tau)c'(t) \rangle_t,
\end{equation}
 and its power spectrum $S_{c}(f)$ is defined as the Fourier Transform of its
autocorrelation function,
\begin{equation}\label{signal_spectrum}
S_{c}(f)  = \mathcal{F}[R_{c}(\tau)].
\end{equation}
 Based on \eqref{eq:EqSintesis}, the following relationships can be derived
between $R_{c}(\tau)$ and $\boldsymbol{\rho}(v,w,\tau)$, and betwen $S_{c}(f)$ and $\boldsymbol{\sigma}(v,w,f)$ (see Appendix A):
\begin{eqnarray}
R_{c}(\tau)&=& \int_{V\times V}{\mathbf{L}^{T}(v) \boldsymbol{\rho}(v,w,\tau) \mathbf{L}(w)}{dvdw},\label{eq:signal_model_autocorr}\\
S_{c}(f) &=&\int_{V\times V}{\mathbf{L}^{T}(v) \boldsymbol{\sigma}(v,w,f) \mathbf{L}(w)}{dv dw}. \label{eq:signal_model_spectrum}
\end{eqnarray}
 Equations \eqref{eq:signal_model_autocorr} and \eqref{eq:signal_model_spectrum}
 reflect the linear relationship between cardiac signals and sources [c.f.
 \eqref{eq:EqSintesis}], and can be used to gain insight into the nature of the
 autocorrelation and spectrum of cardiac signals. According to
 \eqref{eq:signal_model_spectrum}, two factors determine the spectrum of cardiac
 signals. The first factor is the spectrum $\boldsymbol{\sigma}(v,w,f)$ of
 cardiac sources. It is worth noting that this is the solely feature of the
 spatiotemporal dynamics of cardiac sources that manifests on the spectrum of
 cardiac signals. The second factor is the MSD of the lead system,
 $\mathbf{L}(v)$. Since $\mathbf{L}(v)$ is specific for each lead system,
 \eqref{eq:signal_model_spectrum} reveals that cardiac signals measured by
 different lead systems will in general have different spectra for the same
 underlying spatiotemporal dynamics.



\section{M�todos estimaci�n del espectro}
\section{Frecuencia Dominante}

\section{Concluciones}

 