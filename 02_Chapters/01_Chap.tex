% -*-coding: iso-latin-1  -*-
\chapter{Modelos Bioel�ctricos}

\begin{resumen}
  En esta secci�n se abarcan los conceptos b�sicos de anatom�a  y
  fisiolog�a cardiaca, dando un repaso por las arritmias cardiacas, en especial
  la fibrilaci�n. Luego se pasa a dar la visi�n de los modelos num�ricos, como
  los modelos anat�micos de coraz�n en 2D y 3D realistas.
  Luego entramos a analizar los diferentes m�delos para la generaci�n y
  propagaci�n de la actividad el�ctrica del miocardio, se describe el Automata
  celular propuesto en \cite{Alonso-Atienza05}, como un modelo de baja carga
  computacional
  
  Para concluir el capitulo, se hace una introducci�n a los sistema de medida,
  en donde se exponen el volumen equivalente de los diferentes sistemas de electrodos.
\end{resumen} 


\section{Introducci�n}
\subsection{Anatom�a y fisiolog�a de las celular}

algo de ac� http://www.bem.fi/book/, \cite{Malmivuo95}

\section{Modelo Electrofisiol�gico} \label{sec:modelElectrofi}
bibliografia \cite{Sachse04} para el modelado del coraz�n, sin embargo  falta 
ve el modelado de la actividad celular  ejemplo neurociencia

\subsection{Modelo Celular}
\subsection{Modelo de Tejido}

\section{Sistemas de medida}
algo de ac� http://www.bem.fi/book/ \cite{Malmivuo95}

Cardiac sources are the bioelectric processes generated by the heart during
contraction. There exist different, equivalent mathematical paradigms to model
the activity of cardiac sources, such as the monopole field and the dipole
field \cite{Malmivuo95}. In this study, we model cardiac sources as a
time-varying dipole field, i.e. as a spatial distribution of time-varying
dipoles $\mathbf{J}(v,t)= [J_x(v,t), J_y(v,t), J_z(v,t)]^T$ on a volume $V$,
where $v$ denotes a point located inside $V$ and $t$ denotes the time instant.

 The time-varying activity of cardiac sources can be measured by lead systems,
 producing cardiac signals. Taking the dipole field as our reference description
 for cardiac sources, we follow a lead-field approach to model cardiac signals
 \cite{Malmivuo95}. According to the lead-field theory, the cardiac signal
 $c(t)$ that is induced at a lead system by a dipole field $\mathbf{J}(v,t)$ can
 be expressed as

\begin{equation}\label{eq:EqSintesis}
c(t)=\int_V{\mathbf{L}^{T}(v) \mathbf{J}(v,t)}{dv}
\end{equation}

 where the vector field $\mathbf{L}(v)= [L_x(v), L_y(v), L_z(v)]^T$ is the
 measurement sensitivity distribution (MSD) and describes the ability of the
l ead system to measure cardiac dipoles located at $v\in V$. In words, cardiac
si gnals are a weighted linear combination of the underlying cardiac sources.


\subsection{Sistema de Electrodos}
definici�n de las ecuaciones lapalaciano
\subsection{Volumen equivalente de sistema de electrodos}
 \ac{LF}


\section{Conclusiones}

