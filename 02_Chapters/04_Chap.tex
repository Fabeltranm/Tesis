\chapter{Simulaciones}
\begin{resumen}

Este capitulo utilizar el formalismo matemático presentado anteriormente, para analizar en un marco simulado la relación entre el volumen de correlación y el espectro captado por el sistema de electrodos.

Por medio de la implementación de diferentes modelos de la anatomía de tejido cardiaco, y junto a la simulación de diversos sistemas de electrodos, se simulan distintas dinámicas con distintas correlaciones  para ilustrar los resultados analíticos del capítulo anterior.

\commentDir{ ... }

\end{resumen}

\section{Introducción}
% el objetivo es generar dinamicas controladas, con respecto al nivel de
% correlación, ver mapas de correlación

\subsection{Tablas de experimentos}

\section{Implementación de modelos}
\subsection{Modelo de ruido blanco}
\subsection{Modelo Autómata}
\subsubsection{Protocolos de estimulacón}
\subsection{Modelo 2}
\subsection{Modelo 3}
\subsection{Modelo 4}

\section{Modelos de leadfield}

\section{Resolución espacial del sistemas de electrodos}
\section{Relación ancho de banda}


\section{Concluciones}
