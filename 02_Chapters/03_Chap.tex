
\chapter{Análisis}

\begin{resumen}

En este capitulo se realiza el análisis sistemático,  que conecta el espectro de las señales cardiacas con las características espacio temporal de diferentes dinámicas cardiacas, como, ritmos cardiacas altamente organizados, ritmos altamente desorganizados y ritmos parcialmente organizados. Las tres dinámicas de segundo orden,  se modelan  como \acf{FC}, \acf{FI} \acf{FPC}, respectivamente.

El capitulo inicia con la definición  del modelo espaciotemporal \acf{ID}, lo que permite relacionar los modelos de segundo orden con el espectro de las señales adquiridas en arritmias cardiacas. En este sentido se presenta la definición del  espectro de los modelos \ac{FC}, \ac{FI} \ac{FPC} y la respectiva autocorrelación.

A continuación. se presenta la definición y el análisis matemático  de \acf{MSD}, propuesta de  cuantificación de la resolución espacial del sistema de electrodo. Así mismo, se analizan las principales propiedades de la distribución de tiempos de activación, para darle una base sólida en función de la correlación de la  actividad cardiaca.
  

 \commentDir{ ... }

\end{resumen}

\section{Introducción}
\section{Modelos de fuentes de 2 orden}



\subsection{Dinámica Espacio-temporal de \acf{ID}}

Todos los dipolos que conforman la región tienen las mismas características espectrales  y por lo tanto, la misma auto-correlación.   

 \begin{eqnarray}
 \boldsymbol{\rho}(v,v,\tau)=\boldsymbol{\rho}(\tau), \label{corr_IDS}\\
 \boldsymbol{\sigma}(v,v,f)=\boldsymbol{\sigma}(f). \label{spectrum_IDS}
 \end{eqnarray}
 By substituting \eqref{corr_IDS} and \eqref{spectrum_IDS} respectively into \eqref{total_corr} and \eqref{total_spectrum}, it can be proved that the total autocorrelation and total spectrum of all the dipoles are also identical,
 \begin{eqnarray}
 R_{J}(v,v,\tau)&=&R_{J}(\tau)=\mathbf{1}^{T} \boldsymbol{\rho}(\tau)  \mathbf{1},\label{total_corr_IDS}\\
 S_{J}(v,v,f)&=&S_{J}(f)=\mathbf{1}^{T} \boldsymbol{\sigma}(f)  \mathbf{1}. \label{total_spectrum_IDS}
 \end{eqnarray}
 Note that this model only describes the activity of cardiac dipoles individually, and does not specify $\boldsymbol{\rho}(v,w,\tau)$ nor $\boldsymbol{\sigma}(v,w,f)$ for $v\neq w$.

\subsection{Dinámicas Espacio-temporales completamente correlacionadas }

% This model of spatiotemporal dynamics corresponds to highly regular rhythms, such as sinus rhythm, in which the activity of one dipole $\mathbf{J}(w,t)$ can be expressed as a delayed version of the activity of another dipole $\mathbf{J}(v,t)$,
% \begin{equation}\label{def:FCS}
% \mathbf{J}(w,t)=\mathbf{J}(v,t-\zeta (v,w)),
% \end{equation}
% where $\zeta (v,w)$ is defined as the time delay between the activities of dipoles $\mathbf{J}(v,t)$ and $\mathbf{J}(w,t)$. Based on \eqref{def:FCS}, it can be proved (see Appendix B) that FC dynamics are also ID, $\boldsymbol{\rho}(v,v,\tau)=\boldsymbol{\rho}(\tau)$ and $\boldsymbol{\sigma}(v,v,f)=\boldsymbol{\sigma}(f)$ [cf. \eqref{corr_IDS} and \eqref{spectrum_IDS}],  and that the autocorrelation and the spectrum of FC sources can be expressed as:
% \begin{eqnarray}
% \boldsymbol{\rho}(v,w,\tau) &=&\boldsymbol{\rho}(\tau - \zeta (v,w)), \label{eq:autocorrelation_correlated1}\\
% \boldsymbol{\sigma}(v,w,f) &=& \boldsymbol{\sigma}(f) \exp[-j2\pi f \zeta (v,w)]. \label{eq:spectrum_correlated}
% \end{eqnarray}
% Consequently, FC sources are completely characterized by $\boldsymbol{\rho}(\tau)$, $\boldsymbol{\sigma}(f)$ and $\zeta (v,w)$.
%

\subsection{Dinámicas Espacio-temporales completamente Incorrelacionadas}

% This model of spatiotemporal dynamics constitutes an idealization of highly irregular and disorganized rhythms, such as fibrillation, in which there is no second-order relationship between the temporal activity of any pair of cardiac dipoles. The autocorrelation and the spectrum of a FU source are defined as
% \begin{eqnarray}
% \boldsymbol{\rho}(v,w,\tau)=\boldsymbol{\rho}(v,v,\tau)\delta(v-w),\label{eq:autocorrelation_uncorrelated} \\
% \boldsymbol{\sigma}(v,w,\tau)=\boldsymbol{\sigma}(v,v,\tau)\delta(v-w).\label{eq:spectrum_uncorrelated}
% \end{eqnarray}
% In words, the cross-correlation between two cardiac dipoles $\mathbf{J}(w,t)$ and $\mathbf{J}(v,t)$, where $v\neq w$, is null.
%
%

\subsection{Dinámicas Espacio-temporales parcialmente correlacionadas}


\section{Función de distribución de tiempos}



\section{\acf{MSD}}

\subsection{Lead Fields}

\subsubsection{Modelo ideal}

% in this section we define one simple, idealized model of MSD, namely the pulse
% model. The pulse model describes a lead system that measures with the same
% sensitivity every dipole within a region $V_0$ of the volume source $V$, while
% rejecting the rest. Mathematically, this model is defined as
%
% \begin{equation}
% \label{def:pulso}
% \mathbf{L}_{V_0}(v)=
% \begin{cases}
%     \begin{tabular}{ll}
%     $\mathbf{1}$ & if $v\in V_0$\\
%     $\mathbf{0}$ & otherwise
%         \end{tabular}
%  \end{cases}.
% \end{equation}
%
% The pulse model can be treated as an approximation of physical MSD that
% effectively concentrate their measurement in a region $V_0$.
%
% For the subsequent analysis it is also convenient to quantify the spatial
% resolution (SR) of pulse leads. The SR can be defined as the region of the
% cardiac source that contributes the most to the measured signal. In this study
% we quantify the SR of pulse leads by introducing the notion of the lead
% equivalent volume (LEV), which is defined as the relative size of $V_{0}$ to
% the size of $V$,
%
% \begin{equation}
% LEV=\frac{\int_{V_{0}}{dv}}{\int_{V}{dv}}=\frac{M_{V_{0}}}{M_V},\label{eq:SR}
% \end{equation}
% where $M_{V_{0}}$ and $M_V$ are the sizes of $V_{0}$ and $V$ respectively.
% Thus, for local measurements the LEV is close to zero, whereas for global
% measurements where $V_0\simeq V$, the LEV is close to one.
%
%
\section{Resolución Espacial vs Ancho de Banda}

% we identify the relationship between: the spectrum of cardiac signals,
% the spatiotemporal dynamics of cardiac sources and the measurement
% characteristics of lead systems


\section{Concluciones}