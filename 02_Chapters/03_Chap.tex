% -*-coding: iso-latin-1  -*-

\chapter{An�lisis}

\begin{resumen}
 In this section we study analytically the spectral manifestation of two
 second-order models of cardiac sources, namely the fully correlated (FC) source
 and the fully uncorrelated (FU) source.  The FC and FU models are
 physiologically meaningful and can be used to describe the dynamics of,
 respectively, highly organized and highly disorganized cardiac rhythms.

 We firstly define the autocorrelation and the spectrum of the following models
 of spatiotemporal dynamics: identically distributed (ID), FC and FU. The ID
 model is introduced for facilitating the comparison of the spectrum of cardiac
 signals measured during FC and FU dynamics. Secondly, we define a simple,
 idealized model of MSD, namely the pulse model. Because of its simplicity, the
 pulse model is used in the analytical derivations and in the simulation
 experiments throughout this study. Finally, we derive analytically the spectrum
 of cardiac signals measured by pulse MSD during FU and FC dynamics.


\end{resumen}

\section{Introducci�n}
.

\section{Modelos de fuentes de 2 orden}
\subsubsection{Identically distributed spatiotemporal dynamics}
In this model of spatiotemporal dynamics all cardiac dipoles have the same autocorrelation and spectrum, 
\begin{eqnarray}
\boldsymbol{\rho}(v,v,\tau)=\boldsymbol{\rho}(\tau), \label{corr_IDS}\\
\boldsymbol{\sigma}(v,v,f)=\boldsymbol{\sigma}(f). \label{spectrum_IDS}
\end{eqnarray}
By substituting \eqref{corr_IDS} and \eqref{spectrum_IDS} respectively into \eqref{total_corr} and \eqref{total_spectrum}, it can be proved that the total autocorrelation and total spectrum of all the dipoles are also identical, 
\begin{eqnarray}
R_{J}(v,v,\tau)&=&R_{J}(\tau)=\mathbf{1}^{T} \boldsymbol{\rho}(\tau)  \mathbf{1},\label{total_corr_IDS}\\ 
S_{J}(v,v,f)&=&S_{J}(f)=\mathbf{1}^{T} \boldsymbol{\sigma}(f)  \mathbf{1}. \label{total_spectrum_IDS}
\end{eqnarray}
Note that this model only describes the activity of cardiac dipoles individually, and does not specify $\boldsymbol{\rho}(v,w,\tau)$ nor $\boldsymbol{\sigma}(v,w,f)$ for $v\neq w$.

\subsubsection{Fully correlated spatiotemporal dynamics}
This model of spatiotemporal dynamics corresponds to highly regular rhythms, such as sinus rhythm, in which the activity of one dipole $\mathbf{J}(w,t)$ can be expressed as a delayed version of the activity of another dipole $\mathbf{J}(v,t)$,
\begin{equation}\label{def:FCS}
\mathbf{J}(w,t)=\mathbf{J}(v,t-\zeta (v,w)),
\end{equation}
where $\zeta (v,w)$ is defined as the time delay between the activities of dipoles $\mathbf{J}(v,t)$ and $\mathbf{J}(w,t)$. Based on \eqref{def:FCS}, it can be proved (see Appendix B) that FC dynamics are also ID, $\boldsymbol{\rho}(v,v,\tau)=\boldsymbol{\rho}(\tau)$ and $\boldsymbol{\sigma}(v,v,f)=\boldsymbol{\sigma}(f)$ [cf. \eqref{corr_IDS} and \eqref{spectrum_IDS}],  and that the autocorrelation and the spectrum of FC sources can be expressed as:
\begin{eqnarray}
\boldsymbol{\rho}(v,w,\tau) &=&\boldsymbol{\rho}(\tau - \zeta (v,w)), \label{eq:autocorrelation_correlated1}\\
\boldsymbol{\sigma}(v,w,f) &=& \boldsymbol{\sigma}(f) \exp[-j2\pi f \zeta (v,w)]. \label{eq:spectrum_correlated}
\end{eqnarray}
Consequently, FC sources are completely characterized by $\boldsymbol{\rho}(\tau)$, $\boldsymbol{\sigma}(f)$ and $\zeta (v,w)$.


\subsubsection{Fully uncorrelated spatiotemporal dynamics}
This model of spatiotemporal dynamics constitutes an idealization of highly irregular and disorganized rhythms, such as fibrillation, in which there is no second-order relationship between the temporal activity of any pair of cardiac dipoles. The autocorrelation and the spectrum of a FU source are defined as
\begin{eqnarray}
\boldsymbol{\rho}(v,w,\tau)=\boldsymbol{\rho}(v,v,\tau)\delta(v-w),\label{eq:autocorrelation_uncorrelated} \\
\boldsymbol{\sigma}(v,w,\tau)=\boldsymbol{\sigma}(v,v,\tau)\delta(v-w).\label{eq:spectrum_uncorrelated}
\end{eqnarray}
In words, the cross-correlation between two cardiac dipoles $\mathbf{J}(w,t)$ and $\mathbf{J}(v,t)$, where $v\neq w$, is null.


\section{Funci�n de distribuci�n de tiempos}

\section{Lead Fields}

\subsection{Idealized model}

in this section we define one simple, idealized model of MSD, namely the pulse
model. The pulse model describes a lead system that measures with the same
sensitivity every dipole within a region $V_0$ of the volume source $V$, while
rejecting the rest. Mathematically, this model is defined as

\begin{equation}
\label{def:pulso}
\mathbf{L}_{V_0}(v)= 
\begin{cases}
    \begin{tabular}{ll}
    $\mathbf{1}$ & if $v\in V_0$\\
    $\mathbf{0}$ & otherwise
        \end{tabular}
 \end{cases}.
\end{equation}

The pulse model can be treated as an approximation of physical MSD that
effectively concentrate their measurement in a region $V_0$.

For the subsequent analysis it is also convenient to quantify the spatial
resolution (SR) of pulse leads. The SR can be defined as the region of the
cardiac source that contributes the most to the measured signal. In this study
we quantify the SR of pulse leads by introducing the notion of the lead
equivalent volume (LEV), which is defined as the relative size of $V_{0}$ to
the size of $V$,

\begin{equation}
LEV=\frac{\int_{V_{0}}{dv}}{\int_{V}{dv}}=\frac{M_{V_{0}}}{M_V},\label{eq:SR}
\end{equation}
where $M_{V_{0}}$ and $M_V$ are the sizes of $V_{0}$ and $V$ respectively.
Thus, for local measurements the LEV is close to zero, whereas for global
measurements where $V_0\simeq V$, the LEV is close to one.


\section{Resoluci�n Espacial vs Ancho de Banda}
we identify the relationship between: the spectrum of cardiac signals,
the spatiotemporal dynamics of cardiac sources and the measurement
characteristics of lead systems
 