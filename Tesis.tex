% ++++++++++++++++++++++++++++++++++++++++++++++++++++++++++++++++++++++++++++++
% Documento maestros para la elaboración de la tesis  proyectos fin de carrera  y master
%  modificación:	FERNEY BELTRAN
%  Fecha:		    septiembre 2014
% +
%+++++++++++++++++++++++++++++++++++++++++++++++++++++++++++++++++++++++++++++

% Los ficheros necesarios para este documento son:
%
%   Varios/* : 		ficheros con las partes del documento que no
%                   son capítulos ni apéndices (pórtada, agradecimientos,
%                   bibliografia etc.)
%   Capitulos/*.tex : capí­tulos de la tesis
%	Apendices/*.tex:  apéndices de la tesis
%	Config/*:
%			constantes.tex : constantes del documento
%			config.tex : 	 configuración del documento en general y de
%							 las plantillas del mismo
%   tmp/:	usado para los  ficheros temporales de compilación
%
%	se utiliza el entorno de edicion de  eclipse con el  plugin Texlipse
%   para el glosario y los acrónimos se utiliza el paquete glossaries y
%   para su complilacion se utiliza el script en pearl '\makeglossaries'
%	Para que la compilación se realice en eclipse se debe crear el make
%   glossaries en external tool configuration mas información en
%   http://tex.stackexchange.com/questions/45416/using-glossaries-in-texlipse
%.
%	configuración de eclipse :
%	1.)	Incluir en windows ->Preferences -> Texlipse -> Latex Temp File
%		.acn, .acr, .alg, .glo, .xdy , .ist, .gls, .glg
%	2.) Run -> External Tools -> External Tools config .. program
%		en location 			selecionar el path de 	/usr/bin/makeglossaries
%       en working Directory 	${project_loc}/tmp
%		en Argument				-q ${project_name} si el .tex es el mismo nombre
%   3.) Project->Properties->Builders
%		 Selecionar import  y seleccionar la build anteriormente configurado
%
%   nota: en algunos casos no se crea de forma inmediata el pdf o dvi con los
% acronimos. Para solucionar este inconveniente  se puede generar  el build del
% proyecto  2 o 3 veces seguidas. Otra forma es crear en external tolls config
% un nuevo  build para latex y dvipdf igual que el paso 2 y 3 anterior.  de
% esta forma la configuraci
