
\chapter{Introducción}

\begin{resumen}
La presente \nombreDoc tiene como pilar el estudio de los efectos de los
diferentes sistemas de electrodos en el espectro captado de las señales cardíacas, mediante procesado digital de señales y el modelado de diferentes dinámicas cardíacas, en un entornos de simulación.

En este capítulo introductorio, se expone la motivación, el estado del arte, el
objetivo general y los objetivos específicos. Luego, se
aborda la metodología usada para obtener los resultados de la
presente \nombreDoc y, se explica la estructura que sigue el
documento. Para cerrar se exponen las aportaciones de la \nombreDoc.

\TODO{metodologías, aportaciones y objetivos específicos}

\commentDir{ ... }

\end{resumen}


\medskip % Espacio vertical




%%%%%%%%%%%%%%%%%%%%%%%%%%%%%%%%%%%%%%%%%%%%%%%%%%%%%%%%%%%%%%%%%%%%%%%%%%%5

\section{Motivación}
% Las fibrilaciones son una de las arritmias cardíacas de mayor interés y de las
% cuales no se conoce con claridad sus mecanismos de activación. Se ha estudiado
% bastante que una de las causas de las fibrilaciones cardíacas, viene dada por
% zonas del tejido cardíaco que presentan altas frecuencias en la actividad
% cardíaca. Por esto, gracias a la correlación entre la \ac{DF} y la activación de
% los rotores que generan dichos trastornos cardíacos, los mapas de \ac{DF} son
% una técnica muy usada para identificar las zonas potencialmente causantes de la
% fibrilación cardíaca.
%
% En \cite{Berenfeld10}, se da una  aproximación a los mecanismos en fibrilación
% auricular, con los mapas de \ac{DF} intracardíacos, los cuales han sido usados
% como guía para las terapias de fibrilación. En estudios con pacientes,  como se
% observa en \cite{Sanders05, atienza2009, atienza2011, kumagai2013, okumura2012},
% los mapas de \ac{DF} han permitido localizar las zonas cardíacas de alta
% frecuencia durante la \ac{AF}, lo que proporciona una estimación de la tasa de
% activación local del miocardio. De esta manera, es ampliamente aceptado que
% identificar la regiones del tejido cardíaco que presentas mecanismos de
% re-entrada durante la fibrilación cardíaca, es de gran importancia para el
% diagnóstico y el tratamiento de este tipo de arritmias. En otras palabras los
% mapas de \ac{DF}, por medio del análisis espectral, representan una estimación
% de la tasa de activación local del tejido cardíaco.
%
%
% Ahora bien, para los procedimientos de ablación de arritmias cardíacas es
% necesario localizar los sustratos arrítmicos que serán intervenidos, y para ello
% es necesario generar los mapas de \ac{DF} por medio de sistemas invasivos que
% adquieren las señales intracardíaca. CARTO \cite{gepstein1997} es posiblemente
% el sistema de contacto de mayor utilización para realizar el mapeo de arritmias;
% en él, el electrodo registra la actividad eléctrica del sustrato mientras es
% desplazado secuencialmente por el endocardio.
% También se encuentra el mapeo de la actividad eléctrica del corazón sin
% contacto, que consiste en reconstruir la actividad local del endocardio mediante
% múltiples electrodos bipolares, midiendo la actividad eléctrica simultáneamente
% en todos los electrodos. Hay una amplia literatura en donde se discuten y
% presentan métodos para detección y posterior tratamiento de la \ac{AF}, como se
% refleja en \cite{richter2011novel, kogawa2015effect}.
%
%
% El principal inconveniente de los métodos invasivos en la adquisición de señales
% intracardíaca es la hospitalización del paciente, lo que se traduce, en que su
% aplicación se limite a un grupo reducidos de arritmias. Para solventar la
% limitación de los procedimientos invasivos, se están desarrollando técnicas no
% invasivas, como el mapeo de \ac{DF} sobre el torso.  Como se presenta en
% \cite{Richter08, Hsu08,Dibs08}, la base de este procedimiento, es la relación
% entre las señales intracardíacas y el \ac{ECG}.
% Se observa en \cite{Guillem13, Guillem09} que los mapas de \ac{DF} del torso han
% sido comparado con los mapas de \ac{DF} intracardíacos durante una \ac{AF},
% dando resultados muy similares en la identificación de zonas de actividad de
% alta frecuencia. Sin embargo, hasta ahora, no hay una clara definición para la
% relación exacta entre mapas intracardíaca y del torsos.  No se evidencia una
% clara relación entre estos dos procesos.
%
%
% Adicionalmente, los mapas de \ac{DF} han revelado regularidades
% espacio-temporales durante la fibrilación cardíaca, tanto en animales,
% \cite{Skanes98, Mandapati00, Mansour01}, como en humanos \cite{Sanders05,
% Atienza06, atienza2009, Berenfeld11}. Por consiguiente, junto con los mapas de
% \ac{DF}, existe un creciente número de estudios que utilizan técnicas
% espectrales para analizar la fibrilación tanto auricular como ventricular. Sin
% embargo, el significado de la relación entre el espectro de las señales
% cardíacas  y las características espacio-temporales de las dinámicas cardíacas
% sigue siendo confuso a la fecha. A pesar de que las características espectrales
% individuales de señales cardíacas se han relacionado con las características
% espacio-temporales de los ritmos cardíacos \cite{Fischer07, Ng07a, Zlochiver08},
% la relación entre el espectro de señales cardíacas y las características
% espacio-temporales de los ritmos cardíacos no ha sido investigado a fondo. Esto
% hace pensar que los mapeo de \ac{DF} pueden no ser del todo confiables, si no se
% tienen en cuenta los efectos de los sistemas de electrodos.
% De lo anterior, se extrae la importancia de conocer la configuración del sistema
% de electrodos de adquisición, ya que, este puede afectar el espectro de las
% señales cardíacas.
%
% Por lo tanto, con la diversidad de sistemas de adquisición, electrodos
% intracardíacos de contacto y sin contacto, y de superficie corporal, surge la
% pregunta: cómo el espectro captado con diferentes sistemas de electrodos se
% relacionan entre sí?. En este sentido, se destacan dos trabajos  \cite{lemay08}
% y \cite{Requena08}. El trabajo de \cite{lemay08} compara la \ac{SR} de un
% sistema de electrodos ubicado en el torso con otro próximo al tejido cardíaco, y
% concluye, que el electrodo  más próximo al miocardio hace un registro muy local
% de la actividad eléctrica.  Entonces, con el sistema de medida del torso se
% tiene una captura global de la actividad.  Por lo tanto, el mapa de \ac{DF} del
% torso, no es un reflejo directo de los mapas de \ac{DF} intracardiacos, sino mas
% bien el promedio de \ac{SR}. El trabajo realizado por \cite{Requena08}, estudia
% las propiedades de captación de los sistemas de electrodos en desfibriladores
% automáticos implantables y, muestra que las características espectrales de las
% señales cardíacas tienen relación con las propiedades de captación de los
% sistemas de electrodos.
%
% Estos trabajos dan el punto de partida para el desarrollo de esta Tesis
% Doctoral. Por un lado, las discrepancias entre los mapas de \ac{DF}
% intracardíacos y del torso, dan pie para estudiar con mayor profundidad el
% \ac{SR} y proponer un método de cuantificación de \ac{SR} en los sistemas de
% medición de la actividad eléctrica del corazón. Y por otro parte. explicar con
% mayor profundidad la relación entre los espectros de señales cardíacas medidos
% por diferentes sistemas de electrodos, junto con la relación teórica de las
% características espacio-temporales de la actividad eléctrica del corazón. Sin
% duda esto permite mejorar los actuales métodos utilizados en la caracterización
% de las fibrilaciones cardíacas, bien sean por procedimientos invasivos, mapas de
% \ac{DF} intracardíaca, o no invasivos, mapas de \ac{DF} a partir de ECG del
% torso.
%


%%%%%%%%%%%%%%%%%%%%%%%%%%%%%%%%%%%%%%%%%%%%%%%%%%%%%%%%%%%%%%%%%%%%%%%%%%%%%%%%
%%%%%%%%%%%%%%%%%%%%%%%%%%%%%%%%%%%%%%%%%%%%%%%%%%%%%%%%%%%%%%%%%%%%%%%%%%%%%%%%
%%%%%%%%%%%%%%%%%%%%%%%%%%%%%%%%%%%%%%%%%%%%%%%%%%%%%%%%%%%%%%%%%%%%%%%%%%%%%%%%
%%%%%%%%%%%%%%%%%%%%%%%%%%%%%%%%%%%%%%%%%%%%%%%%%%%%%%%%%%%%%%%%%%%%%%%%%%%%%%%%

\section{Estado del Arte}

% El análisis espectral juega un papel importante en la electrofisiología, tanto
% clinica como experimental, es asi como en la electrofisiología cardiaca, los
% métodos espectral han sido ampliamente usado para el estudos de  transtornos
% cardiacos, como la \ac{AF}  \cite{Everett01, Lazar04, Sanders05, Atienza06,
% Atienza09, Uldry12, Guillem13, Kumagai13, Salinet14} y la  \ac{VF}
% \cite{Strohmenger97, Eftestol00, Jekova04, Panfilov09, Mollerus11, Requena13}.
%
%
%
% La técnicas de mapeado intracardiaca combinadas con  el análisis de \ac{DF},
% conocido como mapeo de \ac{DF}, han permiten estudiar las
% características espacio-temporal de las fibrilaciones. Este método, ha
% dejado ver una cierta regularidad espacio-temporal durante un \ac{AF}.
% Estos estudios se han realizado en animales,
% \cite{Skanes98,Mandapati00,Mansour01} y en humanos  en los trabajos
% \cite{Sanders05,Atienza06}. De igual manera, el análisis de los mapas de \ac{DF}
% en una fibrilación auricular en humanos ha permitido identificar las zonas
% anatómicas del miocardio que presentan frecuencias altas
% \cite{ Sanders05, Atienza09, Kumagai13}, los cuales se han planteado como focos
% responsables de la conducción de \ac{AF}.
%
%
% De esta manera, como lo expone Berenfeld en su libro \cite{Berenfeld11Book},
% los mapas de \ac{DF}, han sido una herramienta muy potente para las terapias de
% \ac{AF}. Para los mapas de \ac{DF}, el análisis espectral es la columna
% vertebral para cuantificar el grado de organización de las dinámicas cardiacas
% \cite{Zipes09}. Sin embargo, en la electrofisiología cardiaca, existe una gran
% variedad de sistemas de adquisición, intracardiacos de contacto y sin contactos,
% y de superficie corporal, que miden la actividad eléctrica del corazón, dando
% una visión local, cercana  y distante, respectivamente.
%
% % los sistemas intracardiacos a su vez se clasifican en sistemas de contanto  y
% % sin contacto. En los sistemas de contacto, los electrodos intracardiacos se
% % ubican sobre el endocardio y miden la actividad eléctrica del corazón
% % localmente, mientras que en los sistemas sin contacto, los electrodos
% % intracardiacos no se ubican directamente en el endocardo, proporcionando asi
% % una vista cercana de la actividad electrica del  sustrato cardiaco. Finalmente,
% % los sistemas de adquisición sobre la superficie corporal proporcionan una vista
% % distante de la actividad eléctrica del corazón.
%
% Con una  diversidad de sistemas de electrodos, surgen la pregunta de cómo los
% diferentes sistemas de adquisición se relacionas entre sí y como afectan las
% diferentes medidas en los mapas de \ac{DF}.
%
% En estudios previos, los efectos  de la resolución del sistema de electrodos en
% el espectro de las señales electrofisiológicas han sido analizados con dos
% enfoques diferentes. El primer método utiliza la representación de fuentes
% bioeléctricas como onda en las que se aplica el análisis de Fourier
% espacio-temporal \cite{Nunez95, NunezSrinivasan06b}.
% En el segundo enfoque las fuentes bioeléctricos se modelan como dipolos
% estocástico cuya dinámica se describen en términos estadísticos \cite{Requena08}.
% Más específicamente el análisis realizado por \cite{Requena08} se
% centra en señales cardíacas medidas por sistemas de electrodos idealizadas, para
% dinámicas del tipo bioeléctricas totalmente correlacionados y no correlacionados.
% Ambos enfoques, considerar escenarios idealizados que constan de
% modelos simples de la dinámica cardiaca y del sistema de electrodos para
% demostrado como el espectro de las señales de electrofisiológicas son
% dependientes de \ac{SR} del sistema de electrodos. Así las señales
% de electrofisiología, tienen asociado un tipo de filtro paso bajos o paso alto,
% según sea mayor o menor la escala del sistema de electrodos, respectivamente.
%
%
% Por su parte, en la cuantificación de \ac{SR} de los
% sistemas de electrodos, en la literatura, a partir del trabajo de Rush y
% Driscoll \cite{Rush69}, es ampliamente usada \ac{MSD}.
% Arzbaecher et al. ha investigado la sensibilidad en el corazón
% de electrodos unipolares ubicados las derivadas precordiales y, unipolares y
% bipolares ubicados en la derivada esofágicas \cite{Arzbaecher79}. En
% \cite{Malmivuo97}, se ha propuesto el valor medio del volumen de  sensibilidad,
% como medidad de comparación del \ac{SR} del \ac{EEG} y el magnetoencefalograma.
%
% En \cite{Vaisanen08}, definen la región de sensibilidad del sistema de
% electrodos  como, la relación entre el promedio de las sensibilidad de dos
% regiones, y se utiliza para cuantificar las mediciones de \ac{EEG}.  Por
% último, la resolución del sistema de electrodos fue propuesto para la
% cuantificación la SR del sistema de electrodos en un desfibrilador implantable
% \cite{Requena09}.
%
% \TODO{Hace falta hablar de el estado de arte de los módelos
% cardiacos}
%
%%%%%%%%%%%%%%%%%%%%%%%%%%%%%%%%%%%%%%%%%%%%%%%%%%%%%%%%%%%%%%%%%%%%%%%%%%%%
\section{Objetivos}

% En esta \nombreDoc se investiga de manera sistemática y se desarrolla un
% formalismos matemático para analizar los efecto espectrales de sistemas de
% electrodos arbitrarios sobre distintas dinámica cardiacas, con el fin de
% identificar la conexión entre el espectro de las señales cardiacas  y la
% dinámica espacio-temporal de los ritmos cardíacos, con diversos grados de
% correlación. Teniendo en cuenta,  que esta \nombreDoc sigue el
% enfoque de análisis de señales multivariantes y, que las aportaciones de esta
% \nombreDoc buscan contribuir en mejorar el análisis e interpretación,  con los
% actuales procedimientos invasivos y no invasivos, de los mapas de \ac{DF},
% proponemos los siguientes objetivos específicos:
%
% \begin{itemize}
% \item 1d
% \item 2d
% \item 3d
% \end{itemize}
%

\section{Metodología}

% Esta \nombreDoc, se enmarca en el análisis matemático del espectro, junto con
% los modelos bioeléctricos. así mismo, las simulaciones, se sustentan en la
% implementación numérica de los modelos  bioeléctricos y la estimación espectral
% de las diversas  dinámicas cardiacas.

\section{Aportaciones}

\section{Estructura}

Esta \nombreDoc, se  articula de 4 capítulos, con el fin de conseguir lso
objetivos propuestos, acorde con la metodología de trabajo. Es asi como la
estructura de \nombreDoc esta dada por:


\begin{itemize}
  \item \textit{Capítulo 2}, abarca los conceptos básicos de anatomía  y
  fisiología cardiaca, dando un repaso por las arritmias cardiacas, en especial
  la fibrilación. Luego se pasa a dar la visión de los modelos numéricos, como
  los modelos anatómicos de corazón, a partir del cual se modelan la
  generación y propagación de la actividad eléctrica del miocardio. Para concluir
  el capitulo, se hace una introducción a los sistema de medida, en donde se
  exponen el volumen equivalente de los diferentes sistemas de electrodos.

  \item \textit{Capítulo 3}, aborda la definición del espectro  y los diferentes
  conceptos necesarios para el análisis espectral. la auto-correlación tanto de
  las fuentes cardiacas como de las señales, se trata con el fin de  sentar las
  bases para la comparación espacio-temporal de la actividad eléctrica. Por
  último se presentan, por medios del análisis multivariante, los mecanismos de
  la estimación del espectro, para  general los mapas de frecuencia tanto pico
  como dominante.

  \item \textit{Capítulo 4}. se dedica a presentar el formalismos matemático,
  que conecta el espectro de las señales cardiacas con las características
  espacio temporal de diferentes dinámicas cardiacas. Por lo cual, en esta
  sección se realizar el análisis sistemáticos de \ac{MSD}, nuestra propuesta de
  cuantificación de la resolución espacial del sistema de electrodos. Así mismo,
  se analizan las principales propiedades de la distribución de tiempos de
  activación, para darle una base sólida en función de la correlación de la
  actividad cardiaca.

  \item para culminar  el \textit{Capítulo 5}, utilizar el formalismo matemático
  presentado anteriormente, para analizar en un marco simulado la relación entre
  el volumen de correlación y el espectro captado por el sistema de electrodos.
  Por medio de la implementación de diferentes modelos de la anatomia de tejido
  cardiaco, y junto a la simulación de diversos sistemas de electrodos, se
  simulan distintas dinámicas con distintas correlaciones  para ilustrar los
  resultados analíticos del capítulo anterior.

  \end{itemize}
