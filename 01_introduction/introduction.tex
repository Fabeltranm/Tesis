% -*-coding: iso-latin-1  -*-

\chapter{Introducci�n}

\begin{resumen}
La presente \nombreDoc tiene como pilares el estudio de los efectos  espectrales
en sistemas de electrodos mediante \ldots procesados digital de se�ales
...........
\end{resumen} 


\medskip % Espacio vertical


\section{Motivaci�n}
 \section{Estado del Arte}
\subsection{}
\subsection{}
\subsection{}


el mapeo de \ac{DF} \ldots 
\section{Objetivos}
En esta \nombreDoc se  analiza y simula, el efecto espectral del sistema de
electrodos,(\ac{LF}), en el registro se�ales el�ctricas, \ldots

\section{Metodolog�a}
Esta \nombreDoc, se enmarca en el an�lisis matem�tico del espectro, junto con 
los modelos bioel�ctricos. as� mismo, las simulaciones, se sustentan en la
implementaci�n num�rica de los modelos  bioel�ctricos y la estimaci�n espectral
de las diversas  din�micas cardiacas.

\section{Aportaciones}
\section{Estructura}
Esta \nombreDoc, se  estructura en tres partes, que agrupan 5 cap�tulos. 
la Parte 1 consta de \ldots
la Parte 2 consta de \ldots
la Parte 3 consta de \ldots 
