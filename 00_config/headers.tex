% -*-coding: iso-latin-1  -*-
\usepackage{fancyhdr}			% Cambia los parametros de la cabecera
\pagestyle{fancy}			    % elegir este estilo de cpps
\addtolength{\headheight}{2pt}	% se evitan warnings por el tama�o de la cabecera


\newcommand{\restauraCabecera}{
  \fancyhead[LO]{\rightmark}
  \fancyhead[RE]{\leftmark}
}


\renewcommand{\headrulewidth}{0.5pt}             % Cabecera: subraya la cabecera (fijar en "0pt" si no se desea).
\renewcommand{\footrulewidth}{0pt}               % Pi�: subraya el pie de p�gina (fijar en "0pt" si no se desea).
\renewcommand{\chaptermark}[1]{%
   \markboth{\textsc{\chaptertitlename\ \thechapter.}\ #1}{}%
}
\renewcommand{\sectionmark}[1]{\markright{\thesection.\ #1}}
\fancyhf{}
\restauraCabecera
\fancyhead[RO,LE]{\thepage}
\fancyhead[LE,RO]{\textbf{\thepage}}           % Cabecera: n�mero de p�gina en negrita.
  
% para los cap�tulos que  no tiene numeraci�n  ni secciones  se debe cambiar
\newcommand{\cabeceraEspecial}[1]{
  \fancyhead[LO]{\textsc{#1}}
  \fancyhead[RE]{\textsc{#1}}
}

\newcommand{\Resumen}{Resumen\markright{Resumen}}
%\newcommand{\TocResumen}{\addcontentsline{toc}{section}{Resumen}}


\newcommand{\Conclusiones}{Conclusiones\markright{Conclusiones\ldots}}
%\newcommand{\TocConclusiones}{\addcontentsline{toc}{section}{Conclusiones}}


% Definici�n del estilo en la  p�gina de inicio de cap�tulo: N�mero de
% la p�gina abajo a la derecha, y sin l�nea en la zona superior.
\fancypagestyle{plain}{%
  \fancyhf{}  
  \fancyfoot[R]{\thepage}
  \renewcommand{\headrulewidth}{0pt}
  \renewcommand{\footrulewidth}{0pt}
}

%% Prevenir que al cambiar de capitulo la pagina en blanco no tenga cabecera 
\makeatletter
\def\cleardoublepage{\clearpage\if@twoside \ifodd\c@page\else
  \hbox{}
  \thispagestyle{empty}
  \newpage
  \if@twocolumn\hbox{}\newpage\fi\fi\fi}
\makeatother


% Saber que se esta en modo Debug,

\ifx\release\undefined
  \fancyfoot[LE,RO]{\vspace*{1cm}\small \sc Draft  -- \today}
\fi



