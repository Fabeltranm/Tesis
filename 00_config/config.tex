% si estamos en debug o release
% en modo debug   se debe comentar el release

% \def\release{1}

\ifx\release\undefined
    % modo debug
    \def\debugmode{1}
    \def\printtodos{1}
    \def\printcoments{1}
    \def\nolayoutconfig{1}
    %\psdraft % no aparecen las imágenes en el documento  impreso
    \def\nogeneratoc{1}
   % \def\nogeneraacronimos{1}

\fi



%%%TEXTO

\usepackage{verbatim}		% texto ascii puro tal cual
\usepackage{geometry}		% para la configuración personal de margenes
\usepackage[spanish]{layout}	% depuración de la config de las paginas
				% se incluye el comando \layout en el
				% documento para ver la información de config de pag.

%\usepackage[ansinew]{inputenc}	% windows
\usepackage[T1]{fontenc}		% codificación de las fuentes de  salidas
\usepackage{textcomp}
% \usepackage[latin1]{inputenc}  	% codificación de fuentes
% de entrada \usepackage[utf8]{inputenc}  		% utf8
\usepackage[utf8]{inputenc}
\usepackage{natbib}	% soporte para bibliografía
				% \citep{ref} entre paréntesis
				% \citet{ref} cita normal
				% \citeyear{ref} para la fecha
				% \citeauthor{ref}
				% http://merkel.zoneo.net/Latex/natbib.php


% \usepackage[spanish]{babel}
\usepackage[spanish,activeacute]{babel}	% Soporte para castellano
\fontfamily{garamond}

\usepackage{xcolor}

%% PÁRRAFOS

\usepackage{multicol}		% utilización de multicolumna
\setlength{\parskip}{0.2ex} 	% separación entre párrafos se aumenta
\newcommand{\linespacing}[1]{\renewcommand{\baselinestretch}{#1}\normalsize}

%% TÍTULOS

\usepackage{titlesec} 	% Cambia el formato de los  títulos
			% importante para la definición de encabezados
			% \chaptertitlename, -> \chaptername ("Capítulo")
			% o  a \apendixname ("Apéndice")



% DIRECCIONES WEB

\usepackage{url}		% soporte para incluir dirección. web
				% \url{http://...}
\usepackage{ifpdf}		% para compilaciones con pdflatex
				% en lugar de latex, se define el ifpdf
\ifpdf				% Color y borde de los enlaces
  \RequirePackage[pdfborder=0,colorlinks,hyperindex,pdfpagelabels]{hyperref}
  \def\pdfBorderAttrs{/Border [0 0 0] }
\else
  \usepackage{hyperref}
\fi
\hypersetup{colorlinks=false}


%%  IMAGENES

\usepackage{subfig}		
\usepackage{epsfig}		% imágenes eps
\usepackage{epsf}              % Ciertas manipulaciones a EPSs
% \usepackage[final]{graphicx}
% \usepackage[draft,dvips]{graphicx}
% \usepackage[pdftex]{graphicx}		% para imágenes  JPG PNG PDF

\usepackage{tikz}
\usetikzlibrary{mindmap, trees,backgrounds}

%% TABLAS

\usepackage{tabularx}		% hacer tablas con párrafos
\usepackage{longtable}		% tablas de  varias hojas
\renewcommand{\arraystretch}{1.2}	%(or 1.3) %más espacio entre las lineas de la tabla
\usepackage{colortbl}			% color en la tablas
\usepackage{color}
\usepackage{booktabs} 			% Para poner las tablas elegantes



% MATEMÁTICAS

\usepackage{calc}      % ayuda con cálculos  básicos dentro de latex
\usepackage{amsmath}   % From the American Mathematical Society
\usepackage{amssymb}


%% ACRONIMOS

%% inclusión del paquete glossaries, para  tener los acrónimos
%	xindy 	indica que el motor de  búsqueda es xindy, se necesita del  la instalación de xindy
%	acronym, indica que se genera lista de acrónimos
%	nomumberlist	para  que no incluya la pagina donde aparece el acrónimo
%	sanitize=none evita lso problemas con los acentos  en algunos comandos
%	\ac muestra el acrónimo en su forma  completa la primera vez  que se llama y
%       la forma corta las siguientes 	
%   \ac1 muestra la  forma larga del acrónimo
%	\acs muestra la forma corta y \acf la forma completa
% el fichero fuente esta en bd_acronimos

\usepackage[xindy={language=spanish-traditional},
acronym, nonumberlist,translate=false,toc,
shortcuts,hyperfirst=false,sanitize=none]{glossaries}



%%  OTROS

% Redefinición al español los nombres de las  tablas  e índices, ya que
% el paquete babel  los pone con el nombre de cuadros
\addto\captionsspanish{%
  \def\tablename{Tabla}%
  \def\listtablename{\'Indice de tablas}%
  \def\listfigurename{\'Indice de figuras}
  \def\contentsname{\'Indice general}%
  \def\chaptername{Cap\'itulo}%
  \def\listacronymname{Acrónimos y abreviaturas}%
  \renewcommand*{\acronymname}{Acrónimos y abreviaturas}
  \renewcommand*{\glossaryname}{Glosario}
}


%% GENERACIÓN DE NUEVOS COMANDOS  PARA SIMPLIFICAR LA ESCRITURA

% insertar una figura con el fin de evitar la inclusión de la carperta
% en donde estan las imagenes

%\newcommand{\imagen}[2]{%
%  \includegraphics[#2]{image/#1}%
% }

% insertar figuras directamente desde el texto, se debe indicar 4
% argumentos 1 el nombre del fichero, 2 width=Xcm,height=Ycm,angle=Z,
% 3 la etiqueta de la figura y 4 el titulo de la misma
\newcommand{\figura}[4]{%
  \begin{figure}[t]%
    \begin{center}%
      \imagen{#1}{#2}%
      \caption{#4} \label{#3}%
    \end{center}%
  \end{figure}%
}


% Configuración del resumen de cada capitulo
\newenvironment{resumen}{%
\begin{quotation}\noindent\begin{small}\textbf{\textsc{Resumen:}}%
}%
{%
\end{small}\end{quotation}%
\bigskip%
}

% configuración de comentarios y cosas por hacer

% TODO
\newcommand{\TODO}[1]{{\color{orange} TODO: #1}}
% comentarios Director
\newcommand{\commentDir}[1]{{\color{red} Comentarios Director: #1}}


%%% FIN CONFIG
