% -*-coding: iso-latin-1  -*-
% si estamos en debug o release
 \def\release{1}
% en modo debug  se debe comentar el release 

\ifx\release\undefined
    % modo debug
    \def\printtodos{1}
    \def\printcoments{1}
    %\psdraft % no aparecen las imagenes en el documento  impreso
    \def\comsdestacados{1}
    \def\nogeneratoc{1}
    \def\nogeneraacronimos{1}

\fi



%%%TEXTO

\usepackage{verbatim}		% texto ascii puro tal cual
\usepackage[spanish]{layout}	% depuraci�n de la config de las paginas 
				% se incluye el comando \layout en el 
				% documento para ver la informaci�n de config de pag.

%\usepackage[ansinew]{inputenc}	% windows
\usepackage[T1]{fontenc}		% codificaci�n de las fuentes de  salidas
\usepackage[latin1]{inputenc}  	% codificaci�n de fuentes de entrada
%\usepackage[utf8]{inputenc}  		% utf8

\usepackage[round]{natbib}	% soporte para bibliografia 
				% \citep{ref} entre parentesis
				% \citet{ref} cita normal
				% \citeyear{ref} para la fecha
				% \citeauthor{ref} 
				% http://merkel.zoneo.net/Latex/natbib.php
					

\usepackage[spanish,activeacute]{babel}	% Soporte para castellano


%% PARRAFOS

\usepackage{multicol}		% utilizaci�n de multicolumna
\setlength{\parskip}{0.2ex} 	% separaci�n entre parrafos se aumenta 
\newcommand{\linespacing}[1]{\renewcommand{\baselinestretch}{#1}\normalsize}

%% TITULOS

\usepackage{titlesec} 	% Cambia el formato de lso  titulos  
			% importante para la definicien de encabezados    
			% \chaptertitlename, -> \chaptername ("Capítulo")
			% o  a \apendixname ("Apéndice")


			
% DIRECCIONES WEB

\usepackage{url}		% soporte para incluir direcc. web	
				% \url{http://...} 
\usepackage{ifpdf}		% para compilaciones con pdflatex
				% en lugar de latex, se define el ifpdf
\ifpdf				% Color y borde de los enlaces
  \RequirePackage[pdfborder=0,colorlinks,hyperindex,pdfpagelabels]{hyperref}
  \def\pdfBorderAttrs{/Border [0 0 0] }
\else
  \usepackage{hyperref}
\fi
\hypersetup{colorlinks=false}


%%  IMAGENES

\usepackage{subfig}		% remplaza el obseleto subfigure
\usepackage{epsfig}		% imagenes eps
\usepackage{epsf}              % Ciertas manipulaciones a EPSs
%\usepackage[final]{graphicx}
%\usepackage[draft,dvips]{graphicx}
%\usepackage[pdftex]{graphicx}		% para imagenes  JPG PNG PDF


%% TABLAS

\usepackage{tabularx}		% hacer tablas con p�rrafos
\usepackage{longtable}		% tablas de  varias hojas 
\renewcommand{\arraystretch}{1.2}	%(or 1.3) %m�s espacio entre las lineas de la tabla
\usepackage{colortbl}			% color en la tablas
\usepackage{color}

\usepackage{booktabs} 			% Para poner las tablas elegantes



 
\usepackage{calc} % ayuda con c�lculos  b�sicos dentro de latex


%% ACRONIMOS

%% inclusi�n del paquete glossaries, para  tener lso acronimos
%	xindy 	indicaque el motor de  busqueda es xindy, se necesita del  la instalacion de xindy
%	acronym, indica que se genera lsita de acronimos
%	nomumberlist	para  que no incluya la pagina donde aparece el acronimo
%	sanitize=none evita lso problemas con los acentos  en algunos comandos
%	\ac muestra el acr�nimo en su forma  completa la primera vez  que se llama y
%   la forma corta las siguentes 	\ac1 muestra la  forma larga del acr�nimo
%	\acs muestra la forma corta y \acf la forma completa
% 	el fichero fuente esta en bd_acronimos

\usepackage[xindy={language=spanish-traditional},
acronym, nonumberlist,translate=false,toc,
shortcuts,hyperfirst=false,sanitize=none]{glossaries}



%%  OTROS

% Redefinici�n al espa�ol los nombres de las  tablas  e �ndices, ya que 
% el paquete babel  los pone con el nombre de cuadros
\addto\captionsspanish{%
  \def\tablename{Tabla}%
  \def\listtablename{\'Indice de tablas}%
  \def\listfigurename{\'Indice de figuras}
  \def\contentsname{\'Indice general}%  
  \def\chaptername{Cap\'itulo}%  
  \def\listacronymname{Acr�nimos y abreviaturas}%
  \renewcommand*{\acronymname}{Acr�nimos y abreviaturas}
  \renewcommand*{\glossaryname}{Glosario}
}
 



%% GENERACI�N DE NUEVOS COMANDOS  PARA SIMPLIFICAR LA ESCRITURA

% insertar una figura con el fin de evitar la inclusion de la carperta
% en donde estan las imagenes 
\newcommand{\imagen}[2]{%
  \includegraphics[#2]{image/#1}%
} 

% insertar figuras directamente desde el texto, se debe indicar 4 
% argumentos 1 el nombre del fichero, 2 width=Xcm,height=Ycm,angle=Z,
% 3 la etiqueta de la figura y 4 el titulo de la misma
\newcommand{\figura}[4]{%
  \begin{figure}[t]%
    \begin{center}%
      \imagen{#1}{#2}%
      \caption{#4} \label{#3}%
    \end{center}%
  \end{figure}%
} 


% Configuraci�n del resumen de cada capitulo 
\newenvironment{resumen}{%
\begin{quotation}\noindent\begin{small}\textbf{\textsc{Resumen:}}%
}%
{%
\end{small}\end{quotation}%
\bigskip%
}


%%% FIN CONFIG


