% incluye paquetes para la configuraci�n del documento

ok		\usepackage[spanish,activeacute]{babel}	% Soporte para castellano
ok		%\usepackage[ansinew]{inputenc}	% windows
ok		\usepackage[latin1]{inputenc}  	% unix
ok		%\usepackage[utf8]{inputenc}  		% utf8


\usepackage{setspace}          	% Cambiar el espaciado entre l�neas
\usepackage[spanish]{minitoc}          % Incluir ToCs (�ndice de materias) para cada cap�tulo
\usepackage[absolute]{textpos} 	% Posicionado arbitrario de texto
\usepackage{enumerate}

\usepackage{xspace, amsmath, amssymb}

ok		%  \usepackage[final]{graphicx}
ok		\usepackage[draft,dvips]{graphicx}		% para imagenes  eps es por defecto
ok		%\usepackage[pdftex]{graphicx}		% para imagenes  JPG PNG PDF

ok		\usepackage{subfigure}
\usepackage{psfrag,cite}
ok		\usepackage{epsfig}
ok		\usepackage{epsf}              	% Ciertas manipulaciones a EPSs


ok		\usepackage{booktabs} 			% Para poner las tablas elegantes
ok 		\usepackage{longtable}         	% Tablas ocupen varias p�ginas
ok		\renewcommand{\arraystretch}{1.2}	%(or 1.3) % M�S ESPACIO ENTRE LAS L�NEAS DE UNA TABLA

\usepackage{latexsym} % S�mbolos 


\usepackage{vmargin, pifont, verbatim, amsfonts}
\usepackage{theorem,bm,algorithm,algorithmic}
\usepackage{bm, syntonly,color}



\setpapersize{A4}
\setmarginsrb{1.5in}{1.0in}{1.0in}{1.15in}{\headheight}{\headsep}{0pt}{0pt}



%%%%%%%%%%%%%%%%%%%%%%%%%%%%%%%%%%%%%%%%%%%%%%%%%%%%%%%%%%%%%%
% Different font in captions
%%%%%%%%%%%%%%%%%%%%%%%%%%%%%%%%%%%%%%%%%%%%%%%%%%%%%%%%%%%%%%
\newcommand{\captionfonts}{\small \slshape}%{\footnotesize }

\makeatletter  % Allow the use of @ in command names
\long\def\@makecaption#1#2{%
  \vskip\abovecaptionskip
  \sbox\@tempboxa{{\captionfonts #1: #2}}%
  \ifdim \wd\@tempboxa >\hsize
    {\captionfonts #1: #2\par}
  \else
    \hbox to\hsize{\hfil\box\@tempboxa\hfil}%
  \fi
  \vskip\belowcaptionskip}
\makeatother   % Cancel the effect of \makeatletter
%%%%%%%%%%%%%%%%%%%%%%%%%%%%%%%%%%%%%%%%%%%%%%%%%%%%%%%%%%%%%%

%%%%%%%%%%%%%%%%%%%%%%%%%%%%%%%%%%%%%%%%%%%%%%%%%%%%%%%%%%%%%%
% Preventing figures from appearing on a page by themselves.
%%%%%%%%%%%%%%%%%%%%%%%%%%%%%%%%%%%%%%%%%%%%%%%%%%%%%%%%%%%%%%
\renewcommand{\topfraction}{0.85}
\renewcommand{\textfraction}{0.1}
%%%%%%%%%%%%%%%%%%%%%%%%%%%%%%%%%%%%%%%%%%%%%%%%%%%%%%%%%%%%%%

%%%%%%%%%%%%%%%%%%%%%%%%%%%%%%%%%%%%%%%%%%%%%%%%%%%%%%%%%%%%%%
% Penalizaci�n para viudas y hu�rfanas.
%%%%%%%%%%%%%%%%%%%%%%%%%%%%%%%%%%%%%%%%%%%%%%%%%%%%%%%%%%%%%%
\clubpenalty=10000 
\widowpenalty=10000 
%\displaywidowpenalty=9999
\raggedbottom 
%%%%%%%%%%%%%%%%%%%%%%%%%%%%%%%%%%%%%%%%%%%%%%%%%%%%%%%%%%%%%%


%%%%%%%%%%%%%%%%%%%%%%%%%%%%%%%%%%%%%%%%%%%%%%%%%%%%%%%%%%%%%%
% Aumento la separaci�n entre p�rrafos
%\parskip=3mm
%%%%%%%%%%%%%%%%%%%%%%%%%%%%%%%%%%%%%%%%%%%%%%%%%%%%%%%%%%%%%%
\newcommand{\linespacing}[1]{\renewcommand{\baselinestretch}{#1}\normalsize}



%%%%%%%%%%%%%%%%%%%%%%%%%%%%%%%%%%%%%%%%%%%%%%%%%%%%%%%%%%%%%%%%%%%%%%%%%%%%%%%%
%%Definimos como queremos que se numeren los teoremas, proposiciones, etc.
\newenvironment{review}{}{}
\newtheorem{assumption}{AS\hspace{-0.15cm}}[chapter]
\newtheorem{lemma}{Lema}[chapter]
\newtheorem{proposition}{Proposici�n}[chapter]
\newtheorem{Property}{Propiedad}[chapter]
\newtheorem{observation}{Observation}
\newtheorem{theorem}{Teorema}[chapter]
\newtheorem{Algorithm}{Algoritmo}[chapter]
\newtheorem{corollary}{Corollary}[chapter]
\newtheorem{fact}{Fact}
\newtheorem{remark}{Observaci�n}[chapter]
\newtheorem{test}{Test Case}
\newtheorem{definition}{Definici�n}[chapter]

\def\notation {\vspace{0.5cm}\noindent{\bf Notation:}}
\def\proof    {\vspace{0.1cm}\noindent{\bf Demostraci�n:   }}
\def\myQED {\hfill$\stackrel{\framebox{}}{}$\vspace{0.3cm}}
\newcounter{ctrp}
\newcounter{ctre}
\long\def\symbolfootnote[#1]#2{\begingroup%
\def\thefootnote{\fnsymbol{footnote}}\footnote[#1]{#2}\endgroup}
%%%%%%%%%%%%%%%%%%%%%%%%%%%%%%%%%%%%%%%%%%%%%%%%%%%%%%%%%%%%%%%%%%%%%%%%%%%%%%%%


\linespread{1}




