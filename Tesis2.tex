% ++++++++++++++++++++++++++++++++++++++++++++++++++++++++++++++++++++++++++++++
%  Documento maestros para la elaboraci�n de la tesis  proyectos fin de carrera
%  y master
%  modificaci�n:	FERNEY BELTRAN
%  Fecha:		    septiembre 2012
% ++++++++++++++++++++++++++++++++++++++++++++++++++++++++++++++++++++++++++++++

% Los ficheros necesarios para este documento son:
%
%   Varios/* : 		ficheros con las partes del documento que no
%                   son cap�tulos ni ap�ndices (p�rtada, agradecimientos,
%                   bibliografia etc.) 	
%   Capitulos/*.tex : cap��tulos de la tesis       
%	Apendices/*.tex:  ap�ndices de la tesis       
%	Config/*:			
%			constantes.tex : constantes del documento       
%			config.tex : 	 configuraci�n del documento en general y de  
%							 las plantillas del mismo
%   tmp/:	usado para los  ficheros temporales de compilaci�n
%				
%	se utiliza el entorno de edicion de  eclipse con el  plugin Texlipse
%   para el glosario y los acrónimos se utiliza el paquete glossaries y
%   para su complilacion se utiliza el script de pearl '\makeglossaries'
%	Para que la compilación se realice en eclipse se debe crear el make
%   glossaries en external tool configuration mas información en 
%   http://tex.stackexchange.com/questions/45416/using-glossaries-in-texlipse
%.
%	configuraci�n de eclipse :
%	1.)	Incluir en windows ->Preferences -> Texlipse -> Latex Temp File 
%		.acn, .acr, .alg, .glo, .xdy , .ist, .gls, .glg
%	2.) Run -> External Tools -> External Tools config .. program 
%		en location 			selecionar el path de 	/usr/bin/makeglossaries
%       en working Directory 	${project_loc}/tmp
%		en Argument				-q ${project_name} si el .tex es el mismo nombre 
%   3.) Project->Properties->Builders
%		 Selecionar import  y seleccionar la build anteriormente configurado
%   
%   nota: en algunos casos no se crea de forma inmediata el pdf o dvi con los
% acronimos. Para solucionar este inconveniente  se puede generar  el build del
% proyecto  2 o 3 veces seguidas. Otra forma es crear en external tolls config
% un nuevo  build para latex y dvipdf igual que el paso 2 y 3 anterior.  de
% esta forma la configuración de los builders del proyecto podria quedar de la
% siguiente forma:
% 		latex -> makeglossaries -> latex ->latex -> dvipdf 
%
% La segunda opcion es usar el makefile que acompa�a este documento. ejecutarlo
% desde una terminar o configura  eclipse de la siguiente forma.
% 1 Run -> External Tools -> External Tools config .. program 
%		en location 	selecionar el path de make. Donde esta el script  que
%                       ejecuta el makefile, (en linux lo sabe ejecutando 
%                       whereis make) 
% 2 En working Directory ${project_loc} 	
% 3 En Argument          coloque el que m�s le convenga: latex, pdflatex, 
%                        fastlatex, fastpdf 
% 4  En Project ->Properties -> builders -> selecionar el nmbre anteriormente
% creado y, en la pesta�a build options, seleccionar during builds. 
%
% Para el clean, repetir paso 1 y 2 dle proceso anterior, y en arguments
% colocar clean y en la ventana build options, seleccionar unicamente during
% Clean.
% Para hacer el bakups ejecute Makefile zipBackup
	
\documentclass[12pt,a4paper, twoside ]{book}


%% fichero de configuraci�n
% -*-coding: iso-latin-1  -*-
% si estamos en debug o release
 \def\release{1}
% en modo debug  se debe comentar el release 

\ifx\release\undefined
    % modo debug
    \def\printtodos{1}
    \def\printcoments{1}
    %\psdraft % no aparecen las imagenes en el documento  impreso
    \def\comsdestacados{1}
    \def\nogeneratoc{1}
    \def\nogeneraacronimos{1}

\fi



%%%TEXTO

\usepackage{verbatim}		% texto ascii puro tal cual
\usepackage{geometry}		% para la configuraci�n personal de margenes
\usepackage[spanish]{layout}	% depuraci�n de la config de las paginas 
				% se incluye el comando \layout en el 
				% documento para ver la informaci�n de config de pag.

%\usepackage[ansinew]{inputenc}	% windows
\usepackage[T1]{fontenc}		% codificaci�n de las fuentes de  salidas
\usepackage{textcomp}
\usepackage[latin1]{inputenc}  	% codificaci�n de fuentes
% de entrada \usepackage[utf8]{inputenc}  		% utf8


\usepackage[round]{natbib}	% soporte para bibliografia 
				% \citep{ref} entre parentesis
				% \citet{ref} cita normal
				% \citeyear{ref} para la fecha
				% \citeauthor{ref} 
				% http://merkel.zoneo.net/Latex/natbib.php
					

\usepackage[spanish,activeacute]{babel}	% Soporte para castellano
\fontfamily{garamond}

\usepackage{xcolor}

%% PARRAFOS

\usepackage{multicol}		% utilizaci�n de multicolumna
\setlength{\parskip}{0.2ex} 	% separaci�n entre parrafos se aumenta 
\newcommand{\linespacing}[1]{\renewcommand{\baselinestretch}{#1}\normalsize}

%% TITULOS

\usepackage{titlesec} 	% Cambia el formato de lso  titulos  
			% importante para la definicien de encabezados    
			% \chaptertitlename, -> \chaptername ("Capítulo")
			% o  a \apendixname ("Apéndice")


			
% DIRECCIONES WEB

\usepackage{url}		% soporte para incluir direcc. web	
				% \url{http://...} 
\usepackage{ifpdf}		% para compilaciones con pdflatex
				% en lugar de latex, se define el ifpdf
\ifpdf				% Color y borde de los enlaces
  \RequirePackage[pdfborder=0,colorlinks,hyperindex,pdfpagelabels]{hyperref}
  \def\pdfBorderAttrs{/Border [0 0 0] }
\else
  \usepackage{hyperref}
\fi
\hypersetup{colorlinks=false}


%%  IMAGENES

\usepackage{subfig}		% remplaza el obseleto subfigure
\usepackage{epsfig}		% imagenes eps
\usepackage{epsf}              % Ciertas manipulaciones a EPSs
%\usepackage[final]{graphicx}
%\usepackage[draft,dvips]{graphicx}
%\usepackage[pdftex]{graphicx}		% para imagenes  JPG PNG PDF

\usepackage{tikz}
\usetikzlibrary{mindmap, trees,backgrounds}

%% TABLAS

\usepackage{tabularx}		% hacer tablas con p�rrafos
\usepackage{longtable}		% tablas de  varias hojas 
\renewcommand{\arraystretch}{1.2}	%(or 1.3) %m�s espacio entre las lineas de la tabla
\usepackage{colortbl}			% color en la tablas
\usepackage{color}

\usepackage{booktabs} 			% Para poner las tablas elegantes



% MATEMATICAS
 
\usepackage{calc}      % ayuda con c�lculos  b�sicos dentro de latex
\usepackage{amsmath}   % From the American Mathematical Society
\usepackage{amssymb}


%% ACRONIMOS

%% inclusi�n del paquete glossaries, para  tener lso acronimos
%	xindy 	indicaque el motor de  busqueda es xindy, se necesita del  la instalacion de xindy
%	acronym, indica que se genera lsita de acronimos
%	nomumberlist	para  que no incluya la pagina donde aparece el acronimo
%	sanitize=none evita lso problemas con los acentos  en algunos comandos
%	\ac muestra el acr�nimo en su forma  completa la primera vez  que se llama y
%   la forma corta las siguentes 	\ac1 muestra la  forma larga del acr�nimo
%	\acs muestra la forma corta y \acf la forma completa
% 	el fichero fuente esta en bd_acronimos

\usepackage[xindy={language=spanish-traditional},
acronym, nonumberlist,translate=false,toc,
shortcuts,hyperfirst=false,sanitize=none]{glossaries}



%%  OTROS

% Redefinici�n al espa�ol los nombres de las  tablas  e �ndices, ya que 
% el paquete babel  los pone con el nombre de cuadros
\addto\captionsspanish{%
  \def\tablename{Tabla}%
  \def\listtablename{\'Indice de tablas}%
  \def\listfigurename{\'Indice de figuras}
  \def\contentsname{\'Indice general}%  
  \def\chaptername{Cap\'itulo}%  
  \def\listacronymname{Acrónimos y abreviaturas}%
  \renewcommand*{\acronymname}{Acrónimos y abreviaturas}
  \renewcommand*{\glossaryname}{Glosario}
}
 



%% GENERACI�N DE NUEVOS COMANDOS  PARA SIMPLIFICAR LA ESCRITURA

% insertar una figura con el fin de evitar la inclusion de la carperta
% en donde estan las imagenes 
\newcommand{\imagen}[2]{%
  \includegraphics[#2]{image/#1}%
} 

% insertar figuras directamente desde el texto, se debe indicar 4 
% argumentos 1 el nombre del fichero, 2 width=Xcm,height=Ycm,angle=Z,
% 3 la etiqueta de la figura y 4 el titulo de la misma
\newcommand{\figura}[4]{%
  \begin{figure}[t]%
    \begin{center}%
      \imagen{#1}{#2}%
      \caption{#4} \label{#3}%
    \end{center}%
  \end{figure}%
} 


% Configuraci�n del resumen de cada capitulo 
\newenvironment{resumen}{%
\begin{quotation}\noindent\begin{small}\textbf{\textsc{Resumen:}}%
}%
{%
\end{small}\end{quotation}%
\bigskip%
}

% configiraci�n de comentarios y cosas por hacer 

% TODO
\newcommand{\TODO}[1]{{\color{orange} TODO: #1}}
% comentarios Director
\newcommand{\commentDir}[1]{{\color{red} #1}}


%%% FIN CONFIG




%% Config de cabeceras de capitulos  y secciones especiales
\usepackage{fancyhdr}			% Cambia los parametros de la cabecera
\pagestyle{fancy}			    % elegir este estilo de cpps
\addtolength{\headheight}{2pt}	% se evitan warnings por el tamaño de la cabecera


\newcommand{\restauraCabecera}{
  \fancyhead[LO]{\rightmark}
  \fancyhead[RE]{\leftmark}
}


\renewcommand{\headrulewidth}{0.5pt}             % Cabecera: subraya la cabecera (fijar en "0pt" si no se desea).
\renewcommand{\footrulewidth}{0pt}               % Pié: subraya el pie de página (fijar en "0pt" si no se desea).
\renewcommand{\chaptermark}[1]{%
   \markboth{\textsc{\chaptertitlename\ \thechapter.}\ #1}{}%
}
\renewcommand{\sectionmark}[1]{\markright{\thesection.\ #1}}
\fancyhf{}
\restauraCabecera
\fancyhead[RO,LE]{\thepage}
\fancyhead[LE,RO]{\textbf{\thepage}}           % Cabecera: número de página en negrita.

% para los capítulos que  no tiene numeración  ni secciones  se debe cambiar
\newcommand{\cabeceraEspecial}[1]{
  \fancyhead[LO]{\textsc{#1}}
  \fancyhead[RE]{\textsc{#1}}
}

\newcommand{\Resumen}{Resumen\markright{Resumen}}
%\newcommand{\TocResumen}{\addcontentsline{toc}{section}{Resumen}}


\newcommand{\Conclusiones}{Conclusiones\markright{Conclusiones\ldots}}
%\newcommand{\TocConclusiones}{\addcontentsline{toc}{section}{Conclusiones}}


% Definición del estilo en la  página de inicio de capítulo: Número de
% la página abajo a la derecha, y sin línea en la zona superior.
\fancypagestyle{plain}{%
  \fancyhf{}
  \fancyfoot[R]{\thepage}
  \renewcommand{\headrulewidth}{0pt}
  \renewcommand{\footrulewidth}{0pt}
}

%% Prevenir que al cambiar de capitulo la pagina en blanco no tenga cabecera
\makeatletter
\def\cleardoublepage{\clearpage\if@twoside \ifodd\c@page\else
  \hbox{}
  \thispagestyle{empty}
  \newpage
  \if@twocolumn\hbox{}\newpage\fi\fi\fi}
\makeatother


% Saber que se esta en modo Debug,

\ifx\release\undefined
  \fancyfoot[LE,RO]{\vspace*{1cm}\small \sc Draft  -- \today}
\fi


% Fichero con las macros para crear la bibliografia
%\include{config/config_bib}

 % fichero de las constantes a usar en el documento
%---------------------------------------------------------------------
%
%                          constantes.tex
%
\ifx\release\undefined
  \def\nombreDoc{[DRAFT] Proyecto Tesis Doctoral }
\else
  \def\nombreDoc{Proyecto Tesis Doctoral }
\fi

\def\tituloDoc{ Relating the Spectrum of Cardiac Signals to the Spatiotemporal Dynamics of Cardiac Source }

\def\autorDoc{Ferney Alberto Beltr�n Molina }
\def\directorDoc{Jes�s Requena Carri�n }
\def\universidad{Universidad Rey Juan Carlos }


\ifx\nogeneraacronimos\undefine
	\makeglossaries
	% -*-coding: iso-latin-1  -*-

%
% Fichero con acr�nimos para su uso con Glossaries
%
% elaborado por ferney alberto beltr�n
% fecha de creaci�n	 2 de abril de 2013

% La entrada es:
% \newacronym{hlabeli}{habbrvi}{hfulli}
\newacronym{APD}{APD}{Duraci�n del Potencial de Acci�n}
\newacronym{DI}{DI}{intervalo diast�lico}
\newacronym{CV}{CV}{velocidad de conducci�n}
\newacronym{AP}{AP}{Potencial de Acci�n}
\newacronym{SR}{SR}{Resoluci�n Espacial}
\newacronym{DF}{DF}{Frecuencia Dominante}
\newacronym{PF}{PF}{Frecuencia Pico}
\newacronym{AF}{AF}{Fibrilaci�n Auricular}
\newacronym{VF}{VF}{Fibrilaci�n Ventricular}
\newacronym{LF}{LF}{Lead Field}
\newacronym {AC}{AC}{Aut�mata Celular }
\newacronym{ECG}{ECG}{Electrocardiograma}
\newacronym{EGM}{EGM}{electrogramas}
\newacronym{EEG}{ECG}{Electroencefalograma}
\newacronym{LRd}{LRd}{Luo-Rudy Dynamic}
\newacronym{HRd}{HRd}{Hund-Rudy Dynamic}
 \newacronym{MSD}{MSD}{medida de la distribuci�n de sensibilidad}


\newacronym{DEP}{DEP}{Densidad espectral de potencia}

\newacronym{IEEE}{IEEE}{Institute of Electrical and Electronic Engineers}
\newacronym{FABM}{FABM}{FERNEY ALBERTO BELTRAN MOLINA}

\fi
%
% "Metadatos" para el PDF
%

  \pdfinfo{
   /Author (\autorDoc)
   /Title  (\tituloDoc)
   /CreationDate (\today)
   /Subject (PDFLaTeX)
   /Keywords (PDF;LaTeX)

}

%\usepackage{fancyhdr}			% Cambia los parametros de la cabecera
\pagestyle{fancy}			    % elegir este estilo de cpps
\addtolength{\headheight}{2pt}	% se evitan warnings por el tamaño de la cabecera


\newcommand{\restauraCabecera}{
  \fancyhead[LO]{\rightmark}
  \fancyhead[RE]{\leftmark}
}


\renewcommand{\headrulewidth}{0.5pt}             % Cabecera: subraya la cabecera (fijar en "0pt" si no se desea).
\renewcommand{\footrulewidth}{0pt}               % Pié: subraya el pie de página (fijar en "0pt" si no se desea).
\renewcommand{\chaptermark}[1]{%
   \markboth{\textsc{\chaptertitlename\ \thechapter.}\ #1}{}%
}
\renewcommand{\sectionmark}[1]{\markright{\thesection.\ #1}}
\fancyhf{}
\restauraCabecera
\fancyhead[RO,LE]{\thepage}
\fancyhead[LE,RO]{\textbf{\thepage}}           % Cabecera: número de página en negrita.

% para los capítulos que  no tiene numeración  ni secciones  se debe cambiar
\newcommand{\cabeceraEspecial}[1]{
  \fancyhead[LO]{\textsc{#1}}
  \fancyhead[RE]{\textsc{#1}}
}

\newcommand{\Resumen}{Resumen\markright{Resumen}}
%\newcommand{\TocResumen}{\addcontentsline{toc}{section}{Resumen}}


\newcommand{\Conclusiones}{Conclusiones\markright{Conclusiones\ldots}}
%\newcommand{\TocConclusiones}{\addcontentsline{toc}{section}{Conclusiones}}


% Definición del estilo en la  página de inicio de capítulo: Número de
% la página abajo a la derecha, y sin línea en la zona superior.
\fancypagestyle{plain}{%
  \fancyhf{}
  \fancyfoot[R]{\thepage}
  \renewcommand{\headrulewidth}{0pt}
  \renewcommand{\footrulewidth}{0pt}
}

%% Prevenir que al cambiar de capitulo la pagina en blanco no tenga cabecera
\makeatletter
\def\cleardoublepage{\clearpage\if@twoside \ifodd\c@page\else
  \hbox{}
  \thispagestyle{empty}
  \newpage
  \if@twocolumn\hbox{}\newpage\fi\fi\fi}
\makeatother


% Saber que se esta en modo Debug,

\ifx\release\undefined
  \fancyfoot[LE,RO]{\vspace*{1cm}\small \sc Draft  -- \today}
\fi


%%%%%%%%%%%%%%%%%%%%%%%%%%%%%%%%%%%%%%%%%%%%%%%%%
%-----------------document begins here ---------%
%%%%%%%%%%%%%%%%%%%%%%%%%%%%%%%%%%%%%%%%%%%%%%%%%

\begin{document}


\ifx\release\undefined
  \layout		% para ver la configuraci�n de las pagimas
\fi

% -*-coding: iso-latin-1  -*-

\begin{center}

\thispagestyle{empty} %\vspace{0cm}
\begin{figure}[h]
\centering
	\includegraphics[width=1.3cm]{images/logo_URJC.eps} \label{figure:HS2ant}
	\vspace{-0.5cm}
\end{figure}

\linespacing{1.5}


{\large \universidad} %\vspace{1cm}




 %\rule{.9\textwidth}{0.5pt}

\vfill {\LARGE\bf \nombreDoc }\vspace{0.5cm}

{\Large\bf \textsc \tituloDoc }

% cjbc
%\linespacing{1.2} \vspace{1.5cm}
\vspace{2cm}

{\large Autor:}

{\large\bf \autorDoc}\vspace{1cm}


Director:

{\large\bf Dr. D. \directorDoc}


\vspace{1cm}


\vspace*{1cm} {DEPARTAMENTO DE TEOR�A DE LA SE�AL Y COMUNICACIONES}\vspace{1cm}

\large{Fuenlabrada, junio de 2013}
\end{center}
\vspace{2cm}
\clearpage


\frontmatter % estilo que debe tener el documento (p�gina de t�tulo, tabla de
% contenidos, pr�logos),
	     % N�meros romanos


% -*-coding: iso-latin-1  -*-

\chapter{Resumen}
\cabeceraEspecial{Resumen}
inicio del resumen




\endinput
 
%%% -*-coding: iso-latin-1  -*-

\chapter{Abstract}
\cabeceraEspecial{abstract}
In this Doctoral Dissertation  \ldots\ldots




\endinput

%%% -*-coding: iso-latin-1  -*-

\chapter{Dedicatoria}


\endinput
%%% -*-coding: iso-latin-1  -*-


\chapter{Agradecimientos}


\endinput

%% Tablas de Contenidos
\ifx\nogeneratoc\undefine
  % -*-coding: iso-latin-1  -*-
%config_toc

\setcounter{tocdepth}{2} %  % nivel de subsection en la tabla 
\setcounter{secnumdepth}{3} % nivel de numeraci�n en el documento

 
% Tabla de contenidos.
% \cabeceraEspecial{\'Indice general}

\tableofcontents
\pdfbookmark{Tabla de contenidos}{tabla de contenidos}
 
\newpage 
% �ndice de figuras

\listoffigures
\pdfbookmark[1]{�ndice de figuras}{indice de figuras}

\newpage

% �ndice de tablas

\listoftables
 \pdfbookmark[1]{�ndice de tablas}{indice de tablas}
\newpage
 
\fi
%% Lista de acr�nimos
\ifx\nogeneraacronimos\undefine
     -*-coding: iso-latin-1  -*-
\ifx\nogeneraacronimos\undefine
    
    \setglossarysection{chapter}
    \renewcommand{\glossarymark}[1]{A lo largo de este documento se mantendr�n
   en su forma original aquellos acr�nimos derivados de una expresi�n inglesa
   cuyo uso se encuentre extendido en la literatura cient�fica. 
   
   De acuerdo con las recomendaciones de la Real Academia Espa�ola, en esta
  \tipoDoc los acr�nimos y siglas no se modifican para formar el plural.}
  
  \printglossaries
%    \printglossary[type=\acronymtype]

\if



  
\if


%%---------------- AQU� COMIENZAN LOS CAP�TULOS DEL DOCUMENTO ---------------- %

\mainmatter % estilo que debe tener el texto principal del documento, numeradas
% con n�meros ar��bigos
\restauraCabecera



% -*-coding: iso-latin-1  -*-

\chapter{Introducci�n}

\begin{resumen}
La presente \nombreDoc tiene como pilares el estudio de los efectos  espectrales
en sistemas de electrodos mediante \ldots procesados digital de se�ales
...........
\end{resumen} 


\medskip % Espacio vertical


\section{Motivaci�n y Estado del Arte}
el mapeo de \ac{DF} \ldots 
\section{Objetivos}
En esta \nombreDoc se  analiza y simula, el efecto espectral del sistema de
electrodos,(\ac{LF}), en el registro se�ales el�ctricas, \ldots

\section{Metodolog�a}
Esta \nombreDoc, se enmarca en el an�lisis matem�tico del espectro, junto con 
los modelos bioel�ctricos. as� mismo, las simulaciones, se sustentan en la
implementaci�n num�rica de los modelos  bioel�ctricos y la estimaci�n espectral
de las diversas  din�micas cardiacas.

\section{Aportaciones}
\section{Estructura}
Esta \nombreDoc, se  estructura en tres partes, que agrupan 5 cap�tulos. 
la Parte 1 consta de \ldots
la Parte 2 consta de \ldots
la Parte 3 consta de \ldots 


\part{Nombre parte}
%%\include{02_Chapters/01_Part}
% -*-coding: iso-latin-1  -*-
\chapter{Modelos Bioel�ctricos}

\begin{resumen}

...........
\end{resumen} 

\section{Introducci�n}
\subsection{Anatom�a y fisiolog�a de las celular}

algo de ac� http://www.bem.fi/book/, \cite{Malmivuo95}

\section{Modelo Electrofisiol�gico} \label{sec:modelElectrofi}
bibliografia \cite{Sachse04} para el modelado del coraz�n, sin embargo  falta 
ve el modelado de la actividad celular  ejemplo neurociencia

\subsection{Modelo Celular}
\subsection{Modelo de Tejido}

\section{Sistemas de medida}
algo de ac� http://www.bem.fi/book/ \cite{Malmivuo95}

Cardiac sources are the bioelectric processes generated by the heart during
contraction. There exist different, equivalent mathematical paradigms to model
the activity of cardiac sources, such as the monopole field and the dipole
field \cite{Malmivuo95}. In this study, we model cardiac sources as a
time-varying dipole field, i.e. as a spatial distribution of time-varying
dipoles $\mathbf{J}(v,t)= [J_x(v,t), J_y(v,t), J_z(v,t)]^T$ on a volume $V$,
where $v$ denotes a point located inside $V$ and $t$ denotes the time instant.

 The time-varying activity of cardiac sources can be measured by lead systems,
 producing cardiac signals. Taking the dipole field as our reference description
 for cardiac sources, we follow a lead-field approach to model cardiac signals
 \cite{Malmivuo95}. According to the lead-field theory, the cardiac signal
 $c(t)$ that is induced at a lead system by a dipole field $\mathbf{J}(v,t)$ can
 be expressed as

\begin{equation}\label{eq:EqSintesis}
c(t)=\int_V{\mathbf{L}^{T}(v) \mathbf{J}(v,t)}{dv}
\end{equation}

 where the vector field $\mathbf{L}(v)= [L_x(v), L_y(v), L_z(v)]^T$ is the
 measurement sensitivity distribution (MSD) and describes the ability of the
l ead system to measure cardiac dipoles located at $v\in V$. In words, cardiac
si gnals are a weighted linear combination of the underlying cardiac sources.


\subsection{Sistema de Electrodos}
definici�n de las ecuaciones lapalaciano
\subsection{Volumen equivalente de sistema de electrodos}
 \ac{LF}


\section{Conclusiones}


\part{Nombre parte}
% -*-coding: iso-latin-1  -*-
\chapter{cap2}
\subsection{introducci�n}
d
\subsection{concluciones Cap2}
dd



\newpage



%%-------------------FINAL DE LOS CAP�TULOS DE LA TESIS-------------%


% Ap�ndices
\appendix
% -*-coding: iso-latin-1  -*-
sssss

\backmatter % se usa para el estilo de la parte final del libro (la
% bibliograf�a, los ��ndices de materias).

\addcontentsline{toc}{chapter}{Bibliograf�a} 

 \bibliographystyle{plainnat}
 \begin{small}
   \bibliography{00_varios/biblio}
 \end{small}
 
\end{document}
