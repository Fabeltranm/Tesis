%%%%%%%%%%%%%%%%%%%%%%%%%%%%%%%%%%%%%%%%%%%%%%%%%%%%%%%%%%%%%%%%%%%%%%%%
% Author  : Andrei Sobolevski (April 2009)
% License : Creative Commons attribution license
% Title   : Map of scientific interactions of researchers 
%           affiliated in 2008 to the J.-V. Poncelet laboratory 
%           (UMI 2615 CNRS, http://www.poncelet.ru)
% Notes   : Produced for the 2008 annual report of the lab;
%           layout of subnodes is a result of manual optimization
% Tags    : mindmap, layers
% Submitted to TeXample.net on 16 January 2010

\documentclass[spanish]{book}
\usepackage[utf8]{inputenc}
\usepackage[spanish, es-tabla]{babel}
\usepackage[usenames,dvipsnames]{color}
\usepackage{tikz}
\usetikzlibrary{mindmap, trees,backgrounds}
\usetikzlibrary{snakes}
%\usepackage[paperwidth=25cm,paperheight=22cm,left=1cm,top=1cm]{geometry}
\usepackage[left=1cm,top=1cm]{geometry}

\begin{document}
\pagestyle{empty}
\begin{figure}
  \centering 
	\begin{tikzpicture}[mindmap, text=white,level 1 concept/.append
	style={level distance=120}, connection bar color/.style={every circle connection bar/.append style={
            append after command={[fill=#1]}}}]
 	\tikzstyle{level 1 concept}+=[font=\sf ,level distance=120 ]
  	\tikzstyle{level 2 concept}+=[font=\sf,level distance=80,text=black]
  	\tikzstyle{level 3 concept}+=[font=\sf,text=black ]
  	% INTRODUCCION
	  \begin{scope}[concept color=orange, text=white]
	    \node [concept] at (-9,0){Introducción}
	    [clockwise from=60,concept color=orange!85] 
	    child [grow=25] 
	    	{node [concept] (1_1) {Motivación}}
	    child [grow=-10]
	      	{node [concept] (1_2) {Objetivos}
	      	[clockwise from=-145,concept color=orange!65]
				child [grow=-20] { node[concept] (1_2_1) {obj 1} }
				child [grow=20] { node[concept] (1_2_2) {obj 2} }
				} 
	    child [grow=-115]
	    	{node [concept] (1_3) {Metodología}}
	    child [grow=-45]
	        {node [concept] (1_4) {Aportaciones}}
		child [grow=-80] 
	    	{node [concept] (1_5){Estado de arte} 
	    	[clockwise from=25,concept color=orange!65]
				child [grow=-70]{ node[concept] (1_5_1) { Modelos} }
				child [grow=-110] { node[concept] (1_5_2) {frecuencia Dominante} }
				}
	    ;
	  \end{scope}
	  % METODOS
      \begin{scope}[concept color=red,text=white]
	    \node [concept]  at (-7,-11)  {Modelos Bioeléctricos}[concept color=red!85] 
	    child [grow=25]
	        	{node [concept] (2_1) {Anatomia y Fisiología}}
      		child [grow=60] 
	    	{node [concept] (2_2){Modelo Electrofisiológico} 
	    	[clockwise from=65,concept color=red!65]
				child { node[concept] (2_2_1) {Célula} }
				child [grow=25] { node[concept] (2_2_2) {Tejido} }
				}
	 		child [grow=-5,level distance=150] 
	    	{node [concept] (2_3){Sistema de Medida} 
	    	[clockwise from=40,concept color=red!65,]
				child [grow=-0] { node[concept] (2_3_1) {Sistema de Electrodos} }
				child [grow=50] { node[concept] (2_3_2) {Volumen Equivalente} }
				}
			child [grow=155]
	        	{node [concept] (2_4) {Conclusiones}}		
	      ;
	  \end{scope}
   	  \begin{scope}[concept color=red,text=white]
	    \node [concept]  at (-7,-16)  {Definición del Espectro}[concept color=red!85] 
	    	child [grow=-185] 
	    		{node [concept] (3_1_1) {Señal}}
 	    	child [grow=-150]
	        	{node [concept] (3_1_2) {Vector}}
  	    	child [grow=-115]
	        	{node [concept] (3_1_3) {Campo Vector}}
      		child [grow=-75] 
	    		{node [concept] (3_2){Auto- correlación} 
	    		[clockwise from=20,concept color=red!65]
					child [grow=-0] { node[concept] (3_2_1) {Fuentes} }
					child { node[concept] (3_2_2) {Señal} }
				}
			child [grow=-25]
	        	{node [concept] (3_3) {Estimación del Espectro}
	        	[clockwise from=0,concept color=red!65]
					child { node[concept] (3_3_1) {Pwelch} }
				}		
			child [grow=15]
	        	{node [concept] (3_4) {Mapas de Frecuencia}
	        	[clockwise from=45,concept color=red!65]
					child [grow=25] { node[concept] (3_4_1) {Frecuencia Dominante} }
					child { node[concept] (3_4_2) {Frecuencia Pico} }
				}		
			child [grow=140]
	        	{node [concept] (3_5) {Conclusiones}}		
	      ;
	  \end{scope}
	  
	  \begin{scope}[concept color=green!65!black,text=white]
	    \node [concept] at (5.5,-21) {Análisis} [concept color=green!65!black!75]
	      child [grow=185] 
	        {node [concept] (4_2) {Modelos de fuentes de 2 orden}} 
	      child [grow=150] 
	        {node [concept] (4_3) {Función de  distribución de tiempos}}
	      child [grow=115] 
	        {node [concept] (4_4) {Lead Field}[concept color=green!45!black!50]
				child [grow=110] { node[concept] (4_4_1) {LEV} }
	        }
	      child [grow=80] 
	        {node [concept] (4_5) {Resolución Espacial}};
	  \end{scope}

      \begin{scope}[concept color=blue!65!black,text=white]
	    \node [concept]  at (5,-5)  {Simulaciones}[concept color=blue!65!black!75] 
	    	child [grow=-175]
	        	{node [concept] (5_3) {Modelos Lead Field}}
 	    	child [grow=80]
	        	{node [concept] (5_4) {Resolución Espacial del Sistema de electrodos}}
  	    	child [grow=-140]
	        	{node [concept] (5_5) {Relación Ancho de Banda}}
      		child [grow=115] 
	    		{node [concept] (5_1){Expermentos} 
	    		[clockwise from=20,concept color=blue!65!black!50]
	   	        	child [grow=165] { node[concept] (5_1_1) {Exp 1} }
   		        	child [grow=125] { node[concept] (5_1_2) {Exp 2} }
   	    	    	child [grow= 85] { node[concept] (5_1_3) {Exp 3} }
				}
			child [grow=-80]
	        	{node [concept] (5_2) {Implementa- ción de \\ Modelos}
	        	[clockwise from=0,concept color=blue!65!black!50]
   	        	child [grow=-155] { node[concept] (5_2_1) {Rudo Blanco} }
   	        	child [grow=-110] { node[concept] (5_2_2) {Automata 2D} }
   	        	child [grow= -70] { node[concept] (5_2_3) {3D Realista} }
				}		
			child [grow=150]
	        	{node [concept] (5_6) {Conclusiones}}		
	      ;
	  \end{scope}	  
	  \begin{pgfonlayer}{background}
		\draw [gray!85,circle connection bar,connection bar color=gray!50] 
		(1_4)  to (2_4)
		(1_4)  to (3_5)
		(1_4)  to (5_6)
		(1_5_2) to (2_3)
		(1_5_2) to (3_4)
		(1_2_1) to (5_1_1)
		(1_2_2) to (5_1_2)		
		(1_5_1) to (2_2_1)
		(1_5_1)	to (2_2_2)
 		(2_2)	to (2_1)
	    (3_4)	to (3_3_1)
		(2_3)	to (3_3_1)
		(2_3)	to (4_4_1)
		(4_5)	to (2_3_1)
		(4_3)	to (2_2_2)
		(4_3)	to (2_2_2)
		(4_3)	to (2_2_2)
		(5_2)	to (2_2)
		(5_3)	to (2_3)
		(5_4)	to (2_3_1)
		(5_5)	to (2_3_1)
	
		;
	 \end{pgfonlayer}
	  
	\end{tikzpicture}
  \caption{mapa}

\end{figure}

\end{document}