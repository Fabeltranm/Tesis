% -*-coding: iso-latin-1  -*-

%
% Fichero con acr�nimos para su uso con Glossaries
%
% elaborado por ferney alberto beltr�n
% fecha de creaci�n	 2 de abril de 2013

% La entrada es:
% \newacronym{hlabeli}{habbrvi}{hfulli}
\newacronym{APD}{APD}{Duraci�n del Potencial de Acci�n}
\newacronym{DI}{DI}{intervalo diast�lico}
\newacronym{CV}{CV}{velocidad de conducci�n}
\newacronym{AP}{AP}{Potencial de Acci�n}
\newacronym{SR}{SR}{Resoluci�n Espacial}
\newacronym{DF}{DF}{Frecuencia Dominante}
\newacronym{PF}{PF}{Frecuencia Pico}
\newacronym{AF}{AF}{Fibrilaci�n Auricular}
\newacronym{VF}{VF}{Fibrilaci�n Ventricular}
\newacronym{LF}{LF}{Lead Field}
\newacronym {AC}{AC}{Aut�mata Celular }
\newacronym{ECG}{ECG}{Electrocardiograma}
\newacronym{EGM}{EGM}{electrogramas}
\newacronym{EEG}{ECG}{Electroencefalograma}
\newacronym{LRd}{LRd}{Luo-Rudy Dynamic}
\newacronym{HRd}{HRd}{Hund-Rudy Dynamic}
 \newacronym{MSD}{MSD}{medida de la distribuci�n de sensibilidad}


\newacronym{DEP}{DEP}{Densidad espectral de potencia}

\newacronym{IEEE}{IEEE}{Institute of Electrical and Electronic Engineers}
\newacronym{FABM}{FABM}{FERNEY ALBERTO BELTRAN MOLINA}
